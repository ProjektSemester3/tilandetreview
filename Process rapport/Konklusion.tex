\chapter{Konklusion}

\section{Individuelle konklusioner}
\section{Samlet konklusion}
Ud fra gruppens erfaringer kan det udledes at Scrum gav nogle gavnlige værktøjer til at organisere et projektarbejde. Dog blev disse værktøjer ikke udnyttet optimalt under hele forløbet. I starten af projektarbejdet havde gruppen svært ved at tilvænne sig den nye arbejdsmetode, hvilket medførte til mange misforståelse af Scrum som værktøj. Efter nogle sprints begyndte gruppen at bruge Scrum optimalt, og dette kunne ses på det daglige arbejde.
\\
Efter efterårsferien begyndte gruppen at skrue ned for det ellers høje tempo der havde præget projektarbejdet. Nogle af gruppens medlemmer begyndte at vise mangel på disciplin ved ikke at udføre uddelegerede opgaver til tiden. Den manglende disciplin førte til at gruppen besluttede at køre en mere topstyret ledelsesstruktur. Gruppen begyndte at bevæge sig væk fra Scrum for at skabe mere disciplin i gruppen. Ydermere valgte 2 medlemmer at forlade projektet hvilket satte et stort pres på de resterende medlemmer. Gruppen var nødsaget til at prioritere, velvidende, at ikke alle opgaver kunne nå at løses inden deadline. Disciplinen i gruppen blev gendannet og gruppen begyndte derfra at arbejde mere fokuseret. Gruppen har i løbet af dette forløb følt at product owner rollen har været manglende. Med en veldefineret product owner rolle der kunne sætte en tydelig dagsorden føler teamet at det ville have gået en del bedre. Gruppen har ikke været god til at finde alternative måder hvorpå denne rolle skulle besættes. 