\section{Konflikthåndtering}

I løbet af projektet har der opstået nogle konflikter, som har krævet gruppens fokus. Selvom der i samarbejdskontrakten var lagt et godt fundament for håndtering af konflikter, blev disse værktøjer ikke brugt, da konflikterne opstod. Mange gange blev konflikterne ikke påtalt i håb om at konflikterne kunne løses af sig selv. 

\subsection*{Første konflikt}
Gruppen havde ofte vidt forskellige opfattelser af, hvordan en opgave skulle løses. Mange gange blev beslutningerne taget på stedet, fordi et gruppemedlem var bedre til at argumentere for sit synspunkt end et andet. Dette resulterede i mange forhastede beslutninger, som senere i projektarbejdet skulle laves om. Under udarbejdelse af applikationsmodeller, opstod den første konflikt som blev taget op på gruppebasis. Konflikten bestod i at flere fra gruppen havde forskellige tanker omkring, hvordan et sekvensdiagram skulle udarbejdes. Det betød at der blev udarbejdet et sekvensdiagram, som mange i gruppen var uenige i. Her blev vejleder for første gang brugt til at håndtere en konflikt i gruppen. Der blev aftalt med vejleder at parterne begge skulle udarbejde et forslag til et sekvensdiagram, og der derefter skulle udvælges den rette til projektet. Dette fungerede som en rigtig løsning, da det endelige sekvensdiagram blev udvalgt af hele gruppen, og at der generelt var tilfredshed omkring det.
\subsection{Ambitionsniveau}
En anden konflikt som der ofte opstod i projektarbejdet, var at der i gruppen var forskellige ambitionsniveauer. Ofte havde gruppens medlemmer forskellige løsninger til samme problem, hvor en af løsningerne konstruktionsmæssigt krævede mere tid. Derfor har det ofte været således at gruppen som udgangspunkt valgte den sværeste løsning, for derefter at nedjustere kravene, således at sværhedsgraden og arbejdsbyrden kunne tilpasses gruppen, og den tid der har været til rådighed. Dette problem er der blevet døjet meget med, da der mange gange er blevet brugt tid på meget, som til slut alligevel ikke skulle bruges. 
\subsection{Frafald}
En konflikt som har været meget afgørende for projektet har været, at et af gruppemedlemmerne meldte sig ud af projektarbejdet. Det startede med at gruppemedlemmet undlod at komme til møderne uden afbud. Dette blev der ikke taget hånd om, og der blev ikke holdt nogen samtale for at adressere problemet. Derfor var der i gruppen i en lang periode en forventning om, at gruppemedlemmet ville komme tilbage til gruppearbejdet og fortsætte. Derfor blev arbejdsbyrden ikke nedjusteret i forhold til at gruppen nu kun var på 5 personer. Årsagen til denne konflikt var manglende dialog. Havde gruppen taget dialogen, og fundet ud af hvad der skulle ske, ville gruppen ikke nå til dette problem.
