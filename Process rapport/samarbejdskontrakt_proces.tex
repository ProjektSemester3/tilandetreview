\section{Samarbejdskontrakt}
I starten af projektforløbet blev en samarbejdskontrakt udarbejdet, med henblik på at kunne skabe fællesrammer, som alle gruppens medlemmer var indforstået med.
Gruppen havde alle erfaring med brug af samarbejdskontrakter fra tidligere semestre. Dette betød, at alle havde en god fornemmelse for, hvordan 
samarbejdskontrakten skulle se ud. Under udarbejdelsen af samarbejdskontrakten havde gruppen mange gode og gavnlige diskussioner, som gav gruppen en 
fællesforståelse, af hvorledes gruppearbejdet skulle fungere. 
Efter udarbejdelsen af første udgave af samarbejdskontrakten, blev den ikke længere opdateret i forhold til gruppens arbejdsmetoder. Generelt blev 
samarbejdskontrakten ikke taget i brug ved konflikter. I samarbejdskontrakten indgik der f.eks. et afsnit omkring manglende fremmøde og konsekvenserne deraf. 
Man var som gruppemedlem forpligtet til at melde afbud ved frablivelse og ligeledes meddele gruppen ved forsinkelse. Dette blev overholdt den første periode 
af projektarbejdet, men senere blev dette afsnit ignoreret. I samarbejdskontrakten blev der beskrevet hvilke konsekvenser en sådan forseelse bør munde ud i, 
men dette blev ligeledes heller ikke taget i brug. 
Som konklusion kan man sige at samarbejdskontrakten som udgangspunkt gav gruppen et fælles grundlag for det fremtidige gruppearbejde, men da det var mest
 nødvendigt at bruge samarbejdskontrakten blev den negligeret.
