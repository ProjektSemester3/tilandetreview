\chapter{Projektgennemførelse}
\section{Gruppedannelse}
Allerede inden semesterstart, uge 35, var gruppen dannet. Fælles værdier og enighed om samarbejde på baggrund af erfaringer fra 2. semesters projekt
blev grundlaget for dannelsen af gruppen.

Det har i processen omkring gruppedannelse været diskuteret om gruppen bestod af nok medlemmer ift. det antal gruppemedlemmer ASE anbefaler til et 3. semesters
projekt, nemlig 7-8 personer. På den ene side kan det være fordelagtigt at være en mindre gruppe fordi, der vil være færre synspunkter at tage hensyn til,
og dermed større mulighed for fælles enighed. En ulempe kan være, at arbejdsbyrden per gruppemedlem bliver større. Disse overvejelser indgik i gruppens dannelse.\\

\section{Samarbejdsaftale og arbejdsfordeling}
Gruppens samarbejdsaftale (reference til bilag) har været et godt sted at enes om forventinger og værktøjer, men har også været et dokument under forandring
og et dokument som sjældent har været henvist til. 
Relevansen af samarbejdsaftalen består således først og fremmest i at være enige om visse retningslinjer og krav til hinanden, men også at have et dokument
at henvise til i kraft af uenigheder gruppemedlemmer imellem.

Gruppen har gjort brug af de funktioner, Scrum tilbyder. Hvert gruppemedlem har derfor haft mulighed for selv at vælge og tilrettelægge opgaver. Initiativ,
tillid og selvorganisering har derfor været nøgleord for arbejdsfordelingen ved brug af Scrum.
Indtil implementeringsfasen var der en god fordeling ift. at arbejde med det kendte og det ukendte. F.eks udarbejdede HVB et IBD (reference til ordliste)
selv om det er hardwarearkitektur.

Gruppen bestod overordnet af to teams, et hardwareteam og et software team. I starten af forløbet var det meningen at software gruppen skulle stå for devkit8000
og de tilhørende forbindelser og software såsom device drivere til SPI og GUI. 
Hardware gruppen skulle tage sig af PSoC, og de motorer og sensorer der skulle tilkobles denne. Det viste sig dog, at software på PSoC lå bedre til software gruppen,
og blev derfor overtaget af dem. Det skal dog siges at der hele projektet igennem har været overlapninger, sådan at softwarefolk har arbejdet med hardware og omvendt.

\section{Planlægning}
Planlægning af projektet skete på sprintplanlægningsmøderne. Her blev der for hvert sprint valgt et mål for sprintet, samt hvilke opgaver, det skulle indeholde.
Gruppen havde nogle problemer med at få overblik over projektet samt vurdere hvilke opgaver, der skulle løses først. Der blev lavet en samlet projektplan,
men den blev stort set ikke benyttet.
Der skete mange ændringer undervejs i projektet og flere sprints blev forlænget på grund af manglende tid. En af de store udfordringer var at vurdere omfanget
af forskellige opgaver i et sprint, og derfor endte gruppen ofte med ikke at blive færdige til sprintets afslutning. Det viste sig at en af de sværeste
scrum discipliner var at vurdere opgavernes tidsomfang.

\section{Projektadministration og ledelse}
Projektet blev administeret med scrum-værktøjet agile board, som gav et godt overblik over opgaverne for hvert sprint, og status på disse opgaver
i løbet af sprintet. Opgaverne blev lagt op på sprintplanlægningsmøderne, og hvert gruppemedlem var selv ansvarlig for at skrive sig på opgaver.
Dette var en god måde at uddelegere arbejdet på, og gruppemedlemmerne blev motiveret af at det var meget klart hvem der påtog sig opgaver i gruppen. 
Der blev oprettet en wiki hvori gruppemedlemmerne kunne føre logbog for deres arbejde. Her blev der givet en mere detaljeret beskrivelse af løste opgaver,
samt eventulle problematikker. Disse logbøger var tilgængelige for alle medlemmer i gruppen, hvilket hjalp med at give indsigt i hinandens arbejde.
Github blev brugt til at gemme dokumenter, og holde styr på gruppens filer. Dette gjorde arbejdet med at merge filer nemmere. De fleste gruppemedlemmer
kendte dog ikke til git inden projektet, så dette gav nogle problemer undervejs i projektet.

Det var Scrum Masterens opgave at sørge for at agile boardet blev opdateret, og at gruppemedlemmerne fik skrevet logbøger. Dette viste dig dog at være en krævende
opgave, da disse ting ofte blev overset i projektets travlhed. 
Dette førte også til misforståelser omkring Scrum Master rollen der gjorde, at gruppen blev hierakisk opdelt. Dette blev hurtigt vendt til en flad organisering,
hvor alle gruppemedlemmer var aktive beslutningstagere. På den måde blev ansvaret fordelt og inddragede hele gruppen som gjorde sammenholdet stærkere.

\section{Møder}
Ifølge samarbejdsaftalen skulle gruppen afholde møder to gange ugenligt (mandag og onsdag), med henblik på at lave tværfagligt arbejde samt at plænlægge
de forskellige sprints. Derudover skulle der afholdes korte stå-op møder tre gange ugenligt for at opretholde god kommunikation internt i gruppen,
hvilket er en af scrum metodens varetegn.
Det blev besluttet, at der hver mandag skulle afholdes arbejdsmøde fra 17-21, hvor alle gruppemedlemmer skulle deltage, og dermed sikre at der var
et tæt samarbejde på tværs af faglige kvalifikationer. Grunden til det sene tidspunkt var udfordringer mht. skemaet for de forskellige studieretninger,
samt individulle hensyn heriblandt børneafhentning. 

I starten af projektet formåede gruppen at overholde de aftalte mødetidspunkter, dog blev stå-op møderne ofte uproduktive, og det blev diskuteret en del
hvorledes de kunne gøres mere effektive, og enkelte gruppemedlemmer ønskede flere af disse korte møder for at få bedre kommunikation. Arbejdsmøderne fungerede godt,
dog kom gruppen ofte til at bruge meget tid på sidesprint og diskussioner, der ikke var relevante for projektet.
Problemet med stå-op møderne blev addreseret ved at hver gruppemedlem kort og præcist skulle komme med en mundtlig rapport, hvorefter der hurtigt kom et overblik
over gruppemedlemmernes status. Herudover blev arbejdsmøderne effektiviseret ved at gruppen blev delt op i mindre enheder, som hver fik uddelt en delopgave. 
Gruppen blev hurtigt enige om at arbejdsmøderne fra 17-21 var en dårlig ide, og dette blev hurtigt lavet om således at der blev arbejdet fra 14-18.
Det andet faste arbejdemøde, der lå om onsdagen, blev også hurtigt afskaffet, da der var brug for større fleksibilitet. Der lå ofte laboratorieøvelser
fra andre fag på samme tidspunkt, og derfor blev dette arbejdemøde kun taget i brug, når det var nødvendigt, og når det passede med skoleskemaet.    

\section{Udviklingsforløb}
Der er valgt 3 hovedemner for beskrivelse af udviklingsforløbet. Disse er blevet valgt, da de har haft den største indflydelse på processen, 
både når man kigger på gruppen som helhed, og arbejdsprocessen for projektet.  

\subsection{Scrum}
\subsubsection{Konflikter med ASE-model}
Allerede inden projektets start havde gruppen lagt sig fast på arbejde med scrum, og tilegnet sig de grundlæggende principper for denne udviklingsmetode.
Der blev fra skolens side lagt op til at anvende scrum jf. "vejledning til udviklingsprocessen for semesterprojekt 3 v. 1.3". Dog blev der 3 uger inde
i projektforløbet lagt en fornyet version op af denne vejledning, som ændrede rammerne for fleksibiliteten af udviklingsprocessen. Gruppens opfattelse af 
scrum forløbet skulle derfor op til revision, da scrum i den fornyede vejledning blev påkrævet at følge ASE modellens vandfaldsstruktur. Gruppens opfattelse
af scrum, var en udviklingsmetode med meget frie rammer, hvor hvert sprint var en vidreudvikling af prototypen med dokumentation og implementering som 
parallelt løbende opgaver. Men istedet skulle der følges en struktur hvor ASE-modellens faser blev færdiggjordt sekventielt. Specielt de påkrævede reviews
gjorde det svært at følge scrum, da der skulle tages mange forholdsregler for at kunne nå deadlines for disse reviews.  
 
\subsubsection{Gruppens afvigelser}
Generelt har det været svært at følge scrum, da gruppens forudsætninger ikke rækte til  
\paragraph{Roller}
Product owner rollen blev uddelegeret på hele gruppen, og alle var ansvarlig for oprettelse af issues på backloggen.   
\paragraph{Sprints}
\paragraph{Møder}


\subsubsection{Fordele/ulemper}
\paragraph{Psykologiske effekter}
\paragraph{Kontrol}
\paragraph{Struktur}
\subsubsection{Brug af artifakter}


\subsection{Gruppedynamik}
\subsubsection{Gruppefølelse}
\subsubsection{Ledelse}
\subsubsection{Holdinddeling}
\subsubsection{Effektivitet}
\subsubsection{Strukturering}
\subsubsection{Konflikter}
\paragraph{Fravær}
\paragraph{Engagement}
\paragraph{Udmelding}
\paragraph{Håndtering}

\subsection{Brug af værktøjer}
\subsubsection{Agile board}
\subsubsection{Logbog}
\subsubsection{Github}
\subsubsection{Latex}

I starten af projektet var alle medlemmer meget motiverede og villige til at påtage sig opgaver i projektet. Dog var der for meget fokus på implementeringsfasen,
hvilket satte gruppen bagud i forhold til designfasen.
Hele scrum metoden blev også diskuteret meget, og der var en del misforståelser i starten mht. de forskellige roller m.m. Efter de første par sprints
fik gruppen godt styr på scrum og dens værktøjer, og samarbejdet fungerede godt internt i gruppen.

Midvejs i projektet fungerede stå-op møderne stort set ikke, da gruppemedlemmer, uden at melde afbud, ikke dukkede op til møderne. Stå-op møderne
blev tydeligvis ikke priorieret særlig højt, hvilket også betød at kommunikationen internt i gruppen blev dårligere. Generelt var der på dette tidspunkt i forløbet
dårlig kommunikation, og manglende engagement fra visse gruppemedlemmer, hvilket smittede af på resten af gruppen.
Der var dog stadig en rimelig god fremmøde til arbejdsmøderne om mandagen. Det var på disse møder at der kom lidt samling på gruppen, og et overblik over projektet
kunne skabes. Arbejdsmøderne om onsdagen blev stort set ikke taget i brug længere, hvilket også var grundet i, at gruppen ikke havde så meget brug
for tværfagligt arbejde, som tidligere i projektet.

\section{Konflikthåndtering}

\section{Opnåede erfaringer}

\section{Fremtidigt arbejde}

\section{Scrum}
Fra starten af projektet var gruppen pålagt at følge Scrum-metoden. Denne blev dog fulgt med visse foranstaltninger:

Arbejdsprocessen skulle planlægges iterativt ud fra ASE-modellen, der inddeles i en analyse-, design- implmenterings- og testfase, da det var et krav, at forskellige
arbejdsgrupper skulle lave review af hinandens arbejde. Derfor blev fleksibiliteten af scrum, der, som en agil udviklingsmetode, som sådan ikke følger nogen
fastlagt rutine, nedprioriteret.

Product Owner-rollen overtoges af udviklingsholdet. Der var en vis usikkerhed om, hvem der skulle overtage denne rolle, om det var skolen, vejlederen eller et
enkelt medlem i gruppen. Det blev dog besluttet, at det var gruppen som helhed, der ligeledes fungerede som udviklingshold, der skulle påtage sig dette ansvar, da
det ville være det mest praktiske, eftersom dennes medlemmer havde det bedste kendskab til det produkt, der skulle udvikles, og da ville øge fleksibiliteten af
sprintet, da det ville være gruppen selv, der vurderede, hvad der i Product Backloggen prioriteredes som det væsentligste for produktet på et givent tidspunkt.
At have gruppens medlemmer til på skift af overtage denne rolle under et sprint var en mulighed, der blev diskuteret, men dette ville være upraktisk, da dette,
gruppens størrelse taget i betragtning, ville øge arbejdsbyrden på de resterende medlemmer.

Scrum er egentlig tilegnet software-udvikling, men blev af nødvendighed også benyttet til hardware-udvikling.

Gruppen har gennem projektforløbet tilstræbt at følge de retningslinjer Scrum påbyder:

Sprint Planning-møder blev som regel holdt efter hver afsluttelse på et sprint. I starten af projektet blev der lagt vægt på at undersøge de systemer, vi skulle
arbejde med. Pga. manglende kendskab til Scrum, blev opgaverne her uddelegeret i slutningen af mødet. Denne fremgangsmåde stred imod Scrums princip om, at gruppens
medlemmer selv skulle påtage sig opgaverne i løbet af sprintet, men efter samtale med vejleder blev dette rettet op på. I de efterfølgende sprints tog medlemmerne
selv opgaverne til sig, dog var der alligevel en idé om, hvem der tog hvad pga. gruppeopdelingen. Denne nye fremgangsmåde viste sig at være fordelagtig ift. den gamle,
da det at kunne afslutte én mindre opgave gjorde det enkelte medlem mere motiveret til at påtage sig en ny.
I starten af projektet var der ikke formuleret noget overordnet mål med sprintsne, men op til første review blev det besluttet at formulere et sådant. Dette gjorde
det lettere for gruppen at skabe sig et overblik over formålet med sprintet. Samtidigt blev der påbegyndt en RISK-analyse. Denne er blevet forsøgt fulgt, men
pga. af deadlines for reviews og tilbagefald i arbejdsprocessen, har gruppen måttet omprioritere opgaver efter behov.
