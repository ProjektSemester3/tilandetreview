\chapter{Projektgennemførelse}
\section(Forord}
Dette er procesrapporten for semesterprojekt 3 på retningerne IKT, EE og E ved Aarhus University, School of Engineering. Rapporten er skrevet af gruppe 10, som
har haft Søren Hansen som vejleder. Afleveringsdato for procesrapporten er 20. December 2016, og bedømmelse er den 18. Januar 2017.
Processrapporten indeholder gruppens overvejelser og refleksioner omkring processen for semesterprojektet, og er skrevet på baggrund af de erfaringer
gruppemedlemmerne har gjort sig under projektet.    

\section{Indledning}
Der er i dette semesterprojekt i modsætningen til tidligere blevet lagt meget vægt på proccesen for projektet. Gruppemedlemmerne har derfor været nød til at 
reflektere over hvilke processmæssige udfordringer de er blevet stillet overfor, hvilket har været en ny måde at gribe projektet an på.
Gruppen har især haft fokus på scrum udviklingsmetoden, da det er første gang der bliver stiftet bekendtskab med denne metode i praksis. Denne udviklingsmetode
gav gruppen en række nye værktøjer til styring af projekter, men har på nogle punkter været svær at implementere. I denne procesrapport vil der derfor være et 
særligt fokus på gruppens arbejde med scrum som udviklingsmetode, og hvordan gruppens til- og fravalg af elementer fra scrum påvirkede projektets forløb.
Der vil også blive fokuseret på hvilke konflikter og problemer gruppen stødte på undervejs i projektet. Herunder hvilke beslutninger der blev taget for at 
løse disse udfordringer, samt gruppens refleksioner omkring disse.

\section{Gruppedannelse}
Allerede inden semesterstart, uge 35, var gruppen dannet. Fælles værdier og enighed om samarbejde på baggrund af erfaringer fra 2. semesters projekt
blev grundlaget for dannelsen af gruppen.

Det har i processen omkring gruppedannelse været diskuteret, om gruppen bestod af nok medlemmer ift. det antal gruppemedlemmer ASE anbefaler til et 3. semesters
projekt, nemlig 7-8 personer. På den ene side kan det være fordelagtigt at være en mindre gruppe, fordi der vil være færre synspunkter at tage hensyn til
og dermed større mulighed for fælles enighed. En ulempe kunne være, at arbejdsbyrden per gruppemedlem bliver større. Disse overvejelser indgik i gruppens dannelse.\\

\section{Samarbejdskontrakt}
I starten af projektforløbet blev en samarbejdskontrakt udarbejdet, med henblik på at kunne skabe fællesrammer, som alle gruppens medlemmer var indforstået med.
Gruppen havde alle erfaring med brug af samarbejdskontrakter fra tidligere semestre. Dette betød, at alle havde en god fornemmelse for, hvordan 
samarbejdskontrakten skulle se ud. Under udarbejdelsen af samarbejdskontrakten havde gruppen mange gode og gavnlige diskussioner, som gav gruppen en 
fællesforståelse, af hvorledes gruppearbejdet skulle fungere. 
Efter udarbejdelsen af første udgave af samarbejdskontrakten, blev den ikke længere opdateret i forhold til gruppens arbejdsmetoder. Generelt blev 
samarbejdskontrakten ikke taget i brug ved konflikter. I samarbejdskontrakten indgik der f.eks. et afsnit omkring manglende fremmøde og konsekvenserne deraf. 
Man var som gruppemedlem forpligtet til at melde afbud ved frablivelse og ligeledes meddele gruppen ved forsinkelse. Dette blev overholdt den første periode 
af projektarbejdet, men senere blev dette afsnit ignoreret. I samarbejdskontrakten blev der beskrevet hvilke konsekvenser en sådan forseelse bør munde ud i, 
men dette blev ligeledes heller ikke taget i brug. 
Som konklusion kan man sige at samarbejdskontrakten som udgangspunkt gav gruppen et fælles grundlag for det fremtidige gruppearbejde, men da det var mest
 nødvendigt at bruge samarbejdskontrakten blev den negligeret.
 
 \section{Arbejdsfordeling}
Under projektarbejdet blev der taget en forholdsvis sikker tilgang til produktudviklingen. Til at starte med var der 4 fra IKT retningen, 1 fra E retningen, og 
1 fra EE retningen. Derfor blev det hurtigt besluttet at de 4 fra IKT skulle beskæftige sig med softwareudvikling, mens den resterende gruppe skulle beskæftige 
sig med hardwareudvikling. I løbet af starten af semesteret blev arbejdsfordelingen ændret, da en ud af de fire IKT’ere valgte at skifte retning, og dermed 
sprænge på hardwareudviklingsgruppen. 
I og med at der i projektet blev brugt SCRUM, blev der selvfølgelig også oprettet opgaver, som de enkelte gruppemedlemmer kunne påtage sig. I løbet af de første 
sprint blev opgaverne uddelegeret ved oprettelse. Det vil sige at når en opgave blev oprettet, så blev person som skal løse opgaven også valgt. Efter et par 
sprint hvor der blev kørt med denne form for opgaveuddelegering, blev det besluttet at opgaverne blot skulle op på sprintbackloggen, således at de forskellige 
gruppemedlemmer kunne påtage sig opgaverne løbende. Dette fungerede meget fint, og sprintet hvor dette blev testet var succesfuldt. Senere i projektarbejdet 
følte gruppen, at der var behov for et større overblik, på grund af det store pres gruppen var under. Dette skyldes at en mand meldte sig ud af gruppen. Derfor 
blev det besluttet at opgaverne blev uddelegeret ved oprettelse, som det blev gjort i starten af projektarbejdet.


\section{Planlægning}
Planlægning af projektet skete på sprintplanlægningsmøderne. Her blev der for hvert sprint valgt et mål for sprintet, samt hvilke opgaver, det skulle indeholde.
Gruppen havde nogle problemer med at få overblik over projektet samt vurdere hvilke opgaver, der skulle løses først. Der blev lavet en samlet projektplan,
men den blev stort set ikke benyttet.
Der skete mange ændringer undervejs i projektet og flere sprints blev forlænget på grund af manglende tid. En af de store udfordringer var at vurdere omfanget
af forskellige opgaver i et sprint, og derfor endte gruppen ofte med ikke at blive færdige til sprintets afslutning. Det viste sig at en af de sværeste
scrum discipliner var at vurdere opgavernes tidsomfang. 

\section{Projekt administration og ledelse}
Gruppen blev enige om fra starten at køre med en flad struktur, hvor alle påtog sig opgaven som productowner, og at der dermed ikke var nogen topstyring.  
Det var dog Scrum Masterens opgave at sørge for at agile boardet blev opdateret, og at gruppemedlemmerne fik skrevet logbøger. Han fik dermed en "lederagtig"
position i gruppen, som førte til misforståelser omkring Scrum Master rollen der gjorde, at gruppen blev hierarkisk opdelt. Dette blev hurtigt taget op på et 
retrospektiv, og vendt tilbage til en flad organisering, hvor alle gruppemedlemmer var aktive beslutningstagere. På den måde blev ansvaret fordelt og inddragede hele gruppen 
som gjorde sammenholdet stærkere. Denne flade struktur i gruppen fungerede godt i de første par sprints, og folk var gode til tage initiativ så opgaverne på
agile board blev løst indenfor deadline. Den fejlede dog i sprint 4 og 5, hvor mange af opgaverne på agile board ikke blev taget frivilligt. Nogle af gruppemedlemmerne
slap afsted med ikke at bidrage særlig meget til projektet og det blev tydeligt at gruppen havde brug for en leder, der kunne uddelegere arbejdet
og holde medlemmerne ansvarlige for at løse deres opgave til tiden. Fra sprint 6 blev der derfor udpeget en leder, der skulle sørge for at gruppen kunne 
færdiggøre projektet til tiden. Denne struktur gik dog på kompromis med scrum, hvor teamet ikke er hierarkisk opdelt. Gruppen følte dog at denne ændring var en
nødvendighed for det videre arbejde, da der tydeligvis manglede den fornødne selvdiciplin og ansvarsfølelse til at arbejde med en flad stuktur.

\section{Møder}
Der har fra starten af projektforløbet været et til to ugentlige arbejdsmøder, hvor gruppen tilsidesatte tid til at fokusere på projektarbejde i 2-4 timer. Disse
fandt som regel sted hver onsdag og hver anden mandag. Daily scrum-møder blev afholdt hver mandag, onsdag og fredag. I princippet burde disse have blevet holdt
hver arbejdsdag, men grundet skema-mæssige uoverensstemmelser blandt medlemmerne, valgtes af praktiske årsager denne tredagsordning. Disse foregik ved, at gruppen
mødtes et bestemt sted, hvor medlemmerne snakkede i ca. et kvarter om, hvilke projektrelaterede opgaver, de var i gang med. Der var dog den yderligere undtagelse,
at et enkelt medlem var blevet fritaget for at møde om fredagen grundet deri, at denne havde et frit skema den dag og havde bopæl relativt fjernt fra universitetet.
Det var hensigten, at dette medlem skulle deltage vha. Skype eller Slack, men dette blev aldrig taget i brug. Sprint planning-møder blev som regel holdt om mandagen.
Disse efterfulgte sprint retrospective-møder, som afholdtes med vejleder. Gennem forløbet har der været en svejende fastlæggelse af disse møder. I starten var
tidsplanen nogenlunde som beskrevet ovenfor. I midten af projektet blev gruppen til gengæld ofte set sig nødsat på at udskyde sprint planning-møderne med en dag
eller en uge. Samtidigt blev møderne med vejleder aflyst af denne, så retrospectivet måtte afholdes internt i gruppen.

I starten af projektet formåede gruppen at overholde de aftalte mødetidspunkter, dog blev stå-op møderne ofte uproduktive, og det blev diskuteret en del
hvorledes de kunne gøres mere effektive, og enkelte gruppemedlemmer ønskede flere af disse korte møder for at få bedre kommunikation. Arbejdsmøderne fungerede 
godt, dog kom gruppen ofte til at bruge meget tid på sidespring og diskussioner, der ikke var relevante for projektet.
Problemet med stå-op møderne blev addreseret ved at hver gruppemedlem kort og præcist skulle komme med en mundtlig rapport, hvorefter der hurtigt kom et overblik
over gruppemedlemmernes status. Herudover blev arbejdsmøderne effektiviseret ved at gruppen blev delt op i mindre enheder, som hver fik uddelt en delopgave. 
Gruppen blev hurtigt enige om at arbejdsmøderne fra 17-21 var en dårlig ide, og dette blev hurtigt lavet om således at der blev arbejdet fra 14-18.
Det andet faste arbejdemøde, der lå om onsdagen, blev også hurtigt afskaffet, da der var brug for større fleksibilitet. Der lå ofte laboratorieøvelser
fra andre fag på samme tidspunkt, og derfor blev dette arbejdemøde kun taget i brug, når det var nødvendigt, og når det passede med skoleskemaet.    

\section{Udviklingsforløb}
Der er valgt 3 hovedemner for beskrivelse af udviklingsforløbet. Disse er blevet valgt, da de har haft den største indflydelse på processen, 
både når man kigger på gruppen som helhed, og arbejdsprocessen for projektet.  

\subsection{Scrum}
\subsubsection{Konflikter med ASE-model}
Allerede inden projektets start havde gruppen lagt sig fast på at arbejde med scrum og tilegnet sig de grundlæggende principper for denne udviklingsmetode.
Der blev fra skolens side lagt op til at anvende scrum jf. "Vejledning til udviklingsprocessen for semesterprojekt 3 v. 1.3". Dog blev der 3 uger inde
i projektforløbet lagt en fornyet version op af denne vejledning, som ændrede rammerne for fleksibiliteten af udviklingsprocessen. Gruppens opfattelse af 
scrum-forløbet skulle derfor op til revision, da scrum i den fornyede vejledning blev påkrævet at følge ASE modellens vandfaldsstruktur. Gruppens opfattelse
af scrum var en udviklingsmetode med meget frie rammer, hvor hvert sprint var en videreudvikling af prototypen med dokumentation og implementering som 
parallelt løbende opgaver. I stedet for dette skulle der følges en struktur, hvor ASE-modellens faser blev færdiggjort kronologisk. Specielt de påkrævede reviews
gjorde det svært at følge scrum, da der skulle tages mange forholdsregler for at kunne nå deadlines for disse reviews.
 
\subsubsection{Roller}
Product owner-rollen blev uddelegeret på hele gruppen, og alle var ansvarlige for oprettelse af issues på backloggen. Der var i starten af forløbet en vis
usikkerhed om hvem, der skulle overtage denne rolle, da det medlem, som påtog sig denne, i princippet ikke samtidig kunne være en del af udviklingsholdet.
Det blev af vejleder foreslået, at rollen kunne gå på tur for hvert sprint, men dette, gruppens størrelse og arbejdsfordelingen taget i betragtning, kunne
ikke lade sig gøre, da det ville medføre, at pågældende medlem ikke kunne tage del i arbejdsprocessen. Eftersom der var et praktisk behov for, at hvert medlem
kunne bidrage sit til, at sprintet blev gennemført, blev det besluttet, at gruppen som helhed påtog sig ansvaret som dels product owner og udviklingshold.
I løbet af og efter sprint 5 kom gruppen dog frem til den konklusion, at det kunne have været en fordel med en enkelt product owner, som havde et klart overblik
over, hvad der skulle nås i projektet, da denne kunne have sat nogle klare mål for projektet og hvert sprint, i stedet for den hidtil fulgte orden, hvor gruppens
medlemmer gennem forløbet jævnbyrdigt havde haft mulighed for at påvirke sprintets forløb. Sidstnævnte følger dog scrum-metodens princip om fleksibilitet, men
projektets begrænsninger taget i bratragnint (jf. forrige sektion) har dette i høj grad forvirret gruppen om, hvordan prioritering af opgaver skulle gribes an.
Scrum master-rollen blev påtaget af et enkelt medlem i starten af projektet. Da hverken denne eller andre medlemmer af gruppen havde nogen tidligere erfaring med
scrum, måtte også denne rolle uddelegeres på hele gruppen for at fjerne hele byrden fra det ene medlem. Denne påtog sig dog stadig hovedansvaret med at holde gruppen
obs på især mødeindkaldelser, men det, at resten af gruppen har måttet sætte sig grundigt ind i scrum, har været en konstruktiv læringsoplevelse, som har bidraget
til et positivt arbejdsmilø.

\subsubsection{Sprints}
Gruppen var fra starten af projektet initierede på at følge scrum's sprint-ordning. Allerede før det første sprint begyndte gruppen at tilføje opgaver til product-
backloggen. Disse skulle dække over hele projektets forløb og være så kompakte, at medlemmet, der påtog sig opgaven, vidste præcist, hvornår opgaven var påbegyndt
og afsluttet. Disse tidlige opgaver blev dog fjernet, da de viste sig ikke at blive relevante, da de forholdte sig for overfladisk til deres respektive område.
Hvert efterfølgende sprint indledtes med et sprint planning-møde, hvor gruppen diskuterede, hvad der skulle arbejdes på i løbet af sprintet. På det første sprint
planning-møde begik gruppen de fejl, af mangel på bedre viden: at uddelegere opgaver til medlemmer med det samme; ikke at have et egentligt formuleret mål med
sprintet; at påtage sig en lineær fremgangsmåde til de enkelte issues, således at en lang række opgaver nødvendigvis krævede, at en tidligere var blevet løst.

Efter samtale med vejleder indstillede gruppen sig på, at følge en mere scrum-rettet tilgang. Dette indebar blandt andet at formulere et mål for et sprint,
at lave en risikoanalyse og lade gruppens medlemmer påtage sig opgaver løbende under et sprint. I starten af andet sprint blev gruppen dog klar over, at der
skulle foretages review på visse arkitekturrelaterede emner mm. Dette måtte prioriteres over den frie struktur, scrum tilbyder, da disse reviews, som tidligere
nævnt, var påkrævet for projektets gennemførelse. Det blev dog af gruppen bestemt, at scrum skulle indkapsle de fornødne aspekter af ASE-modellen, således at
der stadig blev opstillet opgaver, som gruppens medlemmer frit kunne påtage sig, og at der blev sat et mål for sprintet: at blive færdig med systemarkitekturen.
Dette fungerede godt. Gruppens samtlige medlemmer oplevede styrken ved scrum: ved på agile-boardet manuelt at afslutte en opgave fik det enkelte medlem
en bekræftelse på, at dennes arbejde gavnede projektets fremgang, og denne følte sig da også motiveret til at påtage sig en ny opgave. Dette sprint var højdepunktet
for gruppens arbejde med scrum. Tredje sprint fortsatte efter samme princip. Dette forløb sig forholdsvist glidende, men ifm. udførslen af applikationsmodellerne,
og muligvis tidspresset af reviewet, blev visse opgaver for dette sprint ikke fyldestgørende udført. Bl.a. interne stridigheder i gruppen samt forvirring omkring,
hvilken retning projektet skulle gå, førte til, at de næste to sprints kulminerede i en total omstrukturering af projektadministrationen. Under disse to sprints
har der været en tendens til, stik imod scrum's principper, at udvide sprintets forløb, påføre sprint-backloggen nye opgaver under sprintet og generelt at ignorere
opdatering af arbjedsproces jf. logbog. Dette kan forklares ved, at der jf. ASE-modellen af gruppen har været et fokus på at nå visse rettelser af systemarkitektur
og at gøre klar til andet review, så der opstod en følelse af, at arbejdsbyrden blev umådeligt forøget ved at holde sig prinipfast til scrums midler. Dette samt
et stærkt behov for at danne sig en ren retningsplan for projektets fremtidige forløb mod projektets deadline, foranledigede gruppen til at sætte scrum til side
til fordel for en mere lineær arbejdsmodel, hvor der blev sat fokus på, at indskrænke projektet til at få lavet en færdig prototype og skrevet rapport og
dokumentation uantastet, at det skred imod scrum.


\subsection{Gruppedynamik}
\subsubsection{Gruppefølelse}
Som nævnt blev gruppen dannet allerede inden semesterstart, og der var en opfattelse af gruppen som "stærk", med engagerede medlemmer der var fagligt dygtige.
Dette gav gruppen et godt udgangspunkt for et vellykket semesterprojekt, hvilket bidrog til en god stemning, og et godt sammenhold. Dette fortsatte i de første
3 sprints, hvor der var god mødediciplin og hvor gruppen var samlet omkring de opgaver der skulle løses. Gruppen mødes jævnligt til stå-op/arbejdsmøderne,
hvilket også bidrog til at at gruppen følte sig samlet omkring de udfordringer vi stod overfor, og der var generelt god kommunikation internt. Dog skete der et 
stort skift i gruppedynamikken efter efterårsferien. Og i de følgende sprints, 4 og 5, opstod der nærmest en opløsning af gruppefølelsen. Kommunikationen begyndte 
at blive dårlig, og visse gruppemedlemmer mødte ikke op til stå-op møderne, hvilket smittede af på de andre medlemmer. Til sidst var gruppen stort set splittet
op i mindre delenheder der hver tog sig af sit eget, og der var ingen følelse af sammenhold. F.eks. arbejdede software gruppen med SPI og GUI, og hardware 
tog sig af sensorer og motorer, uden at der var indbyrdes kommunikation eller overblik over projektet som helhed. Gruppen var dårlig til at håndtere disse 
problemer, og i flere uger blev denne situation uændret, hvilket havde en negativ effekt på gruppefølelsen. Dette blev heller ikke bedre af at et gruppemedlem 
forlod gruppen. Det var først i sprint 5, at disse problemer blev taget op med vejleder, og det blev besluttet at der skulle føres en stram linje mht. mødedeltagelse
på stå-op møderne. Der blev i det hele taget rusket godt op i gruppen af scrum master, og de resterende medlemmer begyndte at komme på rette spor igen. Set i bagspejlet
skulle der klart havde været taget hånd om den manglende gruppefølelse tidligere. Det virker klart at efterårsferien var med til at stoppe den ellers
gode udvikling gruppen var i. I fremtiden vil det være godt at sikre sig at en pause i skolegangen ikke bliver en hindring for projektet, og man kunne evt. 
have fortsat kommunikationen hen over ferien, og givet medlemmerne nogle lektier for. Desuden ville det måske være godt at holde et møde straks efter ferien
for at sikre sig, at man har et godt udgangspunkt for det fortsatte arbejde.  

\subsubsection{Holdinddeling}
Der var i de første sprints en klar opdeling i teamet. Det blev tydeligt i diskussioner når holdninger blev ytret. Disse holdninger var især metode- og procesrelateret, men strakte 
sig ind i ligegyldige diskussioner om irrelevante emner der ikke nødvendigvis havde noget med projektet at gøre. Det var desværre med til at polarisere klikerne som påvirkede 
arbejdsgangen. Disse kliker må siges at have været et problem, men med en rettidig indgriben kunne meget have været undgået. Efter det sjette medlem meldte sig ud af gruppen 
skete der en opblødning af klikerne og teamet begyndte at arbejde fokuseret og fik et bedre sammenhold.

\subsubsection{Effektivitet og struktur}
For gruppen har struktur gennem hele forløbet været afgørende for effektiviteten. Der har gennem store dele af processen været skift mellem en blød topstyring og en direkte
flad organisering. Det har ofte resulteret i en afventende holdning i teamet hvor ansvarsfølelsen er udeblevet. Dog har det ført til diskussioner om løsninger på problemerme der 
er opstået, og har derigennem fået gruppen til at arbejde mod bedre vilkår. I det store hele er det lykkedes gruppen at bevæge sig fremad og udvikle sig til det bedre. 
Med en konstant revurdering af sig selv har gruppen formået at forstå hinanden og processen markant bedre.
Et stort problem har været den flade struktur uden en ekstern product owner. Uden eksterne krav havde gruppen kun ansvar for sig selv hvilket resulterede i en doven og passiv 
holdning til projektet. Det sås bl.a. under sprint 4 og 5 hvor sprintene blev udskudt fordi opgaverne ikke var blevet løst indenfor den normerede tid. I stedet for at afslutte 
sprintene, holde et grundigt analyserende og diskuterende retrospektiv og starte på en frisk, blev sprintene forlængede med det argument at det var samme opgaver som skulle løses. 
Denne beslutning resulterede dog i et mistet overblik og en demotiverende udvikling samt et meget lavt effektivitetsniveau. 
Under denne fase fik gruppen en endnu større udfordring - et medlem valgte at melde sig ud af studiet og dermed gruppen. Det efterlod 5 mand til at gennemføre et projekt der i 
forvejen syntes ambitionsfyldt. Det giver anledning til at tro at gruppen faldt yderligere fra hinanden, men der skete en opblomstring mellem medlemmerne. Det samlede gruppen og 
der blev gjort tiltag for at effektivisere - deadlines/sprints på under en uge ad gangen og hårdere topstyring. Samtidig blev det aftalt at et medlem ville blive noteret hvis 
denne ikke havde færdiggjort opgaver, som personen var blevet pålagt, og vedlagt i den endelige aflevering. Tiltagene øgede motivationen og havde den ønskede effekt.

Nedenstående er ikke tænkt skal være med:
I starten af projektet blev gruppen motiveret af tanken om semesterstart, nye udfordringer og fællesskabsfølelse. Af denne grund var gruppen både ambitiøs og effektiv
Disse følelser varede indtil ugen før efterårsferien.
Ineffektivitet -> konflikt -> dobbeltarbejde
Sammenlign sprints og forklar forskelle, fordele/ulemper.
Flad organisering var dårligt for gruppen fordi ingen tog teten. Manglede en ekstern product owner til at motivere da der ville være krav at skulle leve op til. 
God effektivitet var afhængig af fællesskabsfølelse, men i højere grad af lederskab og struktur.
Lange sprints blev uoverskuelige og medførte íneffektivitet. 
Højest effektivitet i små grupper af 2 mand eller alene. Første fase af implementering gik godt med stort set al HW og SW færdigt. 

\subsection{Konflikter}
\subsubsection{Konflikthåndtering}
I løbet af projektet har der opstået nogle konflikter, som har krævet gruppens fokus. Selvom der i samarbejdskontrakten var lagt et godt fundament for 
håndtering af konflikter, blev disse værktøjer ikke brugt, da konflikterne opstod. Mange gange blev konflikterne ikke påtalt i håb om at konflikterne kunne 
løses af sig selv. 

\subsubsection{Første konflikt}
Gruppen havde ofte vidt forskellige opfattelser af, hvordan en opgave skulle løses. Mange gange blev beslutningerne taget på stedet, fordi et gruppemedlem var 
bedre til at argumentere for sit synspunkt end et andet. Dette resulterede i mange forhastede beslutninger, som senere i projektarbejdet skulle laves om. 
Under udarbejdelse af applikationsmodeller, opstod den første konflikt som blev taget op på gruppebasis. Konflikten bestod i at flere fra gruppen havde 
forskellige tanker omkring, hvordan et sekvensdiagram skulle udarbejdes. Det betød at der blev udarbejdet et sekvensdiagram, som mange i gruppen var uenige i. 
Her blev vejleder for første gang brugt til at håndtere en konflikt i gruppen. Der blev aftalt med vejleder at parterne begge skulle udarbejde et forslag til 
et sekvensdiagram, og der derefter skulle udvælges den rette til projektet. Dette fungerede som en rigtig løsning, da det endelige sekvensdiagram blev udvalgt 
af hele gruppen, og at der generelt var tilfredshed omkring det.
\subsubsection{Ambitionsniveau}
En anden konflikt som der ofte opstod i projektarbejdet, var at der i gruppen var forskellige ambitionsniveauer. Ofte havde gruppens medlemmer forskellige 
løsninger til samme problem, hvor en af løsningerne konstruktionsmæssigt krævede mere tid. Derfor har det ofte været således at gruppen som udgangspunkt 
valgte den sværeste løsning, for derefter at nedjustere kravene, således at sværhedsgraden og arbejdsbyrden kunne tilpasses gruppen, og den tid der har været 
til rådighed. Dette problem er der blevet døjet meget med, da der mange gange er blevet brugt tid på meget, som til slut alligevel ikke skulle bruges. 
\subsubsection{Frafald}
En konflikt som har været meget afgørende for projektet har været, at et af gruppemedlemmerne meldte sig ud af projektarbejdet. Det startede med at 
gruppemedlemmet undlod at komme til møderne uden afbud. Dette blev der ikke taget hånd om, og der blev ikke holdt nogen samtale for at adressere problemet. 
Derfor var der i gruppen i en lang periode en forventning om, at gruppemedlemmet ville komme tilbage til gruppearbejdet og fortsætte. Derfor blev 
arbejdsbyrden ikke nedjusteret i forhold til at gruppen nu kun var på 5 personer. Årsagen til denne konflikt var manglende dialog. Havde gruppen taget 
dialogen, og fundet ud af hvad der skulle ske, ville gruppen ikke nå til dette problem.

\subsection{Brug af værktøjer}
\subsubsection{Agile board}
Projektet blev administreret med scrum-værktøjet agile board, som gav et godt overblik over opgaverne for hvert sprint, og status på disse opgaver
i løbet af sprintet. Opgaverne blev lagt op på på agile boardet ved hvert sprintplanlægningsmøde, og hvert gruppemedlem var selv ansvarlig for at skrive sig på
opgaver. Dette var en god måde at uddelegere arbejdet på, og gruppemedlemmerne blev motiveret af at det var meget klart hvem der påtog sig opgaver i gruppen.
Dette fungerede rigtig godt i de første 3 sprints, hvor begejstringen for dette nye værktøj også var tydeligt i gruppen. Gruppen fik på disse sprints, de 
fleste opgaver løst, og boardet blev brugt flittigt. Men i de senere sprints blev agile board brugt mindre og mindre, og til sidste stod gruppen i den situation
at boardet stort set ikke blev opdateret. Der var mange opgaver som var løst, men som ikke var blevet opdateret. Dette gav et forkert billede af 
gruppens status, og bidog til forvirring og en følelse af uoverskuelighed. Det endte også med at sprint 4 og 5 blev forlænget flere gange, da der var mange 
uløste opgaver på agile board, og der var ikke overskud til at få ryddet op i opgaverne og starte et nyt sprint. Til sidst i sprint 5 blev det dog for uoverskueligt
og det blev besluttet at få ryddet agile board og starte på en frisk. Det gav gruppen det nødvendige overblik og fokus der skulle til for at komme på rette
spor igen. Gruppen lærte at Agile board på mange måder giver et godt billede og gruppens situation. Et struktureret agile board, som hele tiden bliver opdateret 
er meget vigtig for et godt procesforløb. Et dårligt overblik på agile board kan endda være med til at demotivere gruppen, fordi man mister overblikket.  
 
\subsubsection{Logbog}
Et andet værktøj der blev brugt var oprettelsen af en wiki hvori gruppemedlemmerne kunne føre logbog for deres arbejde. Her blev der givet en mere detaljeret 
beskrivelse af løste opgaver, samt eventuelle problematikker. Disse logbøger var tilgængelige for alle medlemmer i gruppen, hvilket hjalp med at give indsigt i 
hinandens arbejde. Disse logbøger blev også brugt flittigt i starten af projektet, men knap så meget senere. Enkelte gruppemedlemmer holdte helt op med at 
føre logbog, og der blev ikke tjekket op på det af scrum master. Gruppen havde mistet overskuddet til at tage sig af disse "småting", og det blev derfor
nedprioriteret, dog uden den helt store konsekvens for projektet.

\subsubsection{Github}
Github blev brugt til at gemme dokumenter, og holde styr på gruppens filer. De fleste gruppemedlemmer kendte dog ikke til git inden projektet, 
så dette gav nogle problemer især i starten af projektet med bl.a at merge filer. Overordnet var det et godt værktøj, der gjorde at gruppen for det meste havde
et overblik over hvilke dokumenter der var blevet færdiggjort. De enkelte medlemmer kunne nemt klone de forskellige reporsitories, og på den måde undgik gruppen
at det lå filer lokalt på medlemmers PCer, alle dokumenter var altid tilgængelige for alle.  

\subsubsection{Latex}
Enkelte medlemmer i gruppen havde et stort ønske om at anvende Latex til rapportskrivning. De havde i tidligere semesterprojekter brugt meget tid på at
samle dokumenter med word, og Latexs "include" funktion skulle gøre dette meget nemmere. Desuden er Latex et værktøj som anvendes meget på tekniske studier
og det ville derfor gavne hele gruppen at få kendskab til Latex. Det viste sig dog at Latex blev en af de store udfordringer, da det kræver en del
tid at sætte sig ind i hvordan det virker, noget som for de fleste gruppemedlemmer ikke blev prioriteret særlig højt. Dette betød at der kun var et medlem som havde 
styr på latex, mens de andre var i vildrede og derfor var afhængige af at dette ene medlem tog sig af de problemer der opstod. Dette var naturligvis ikke 
optimalt, og gav frustrationer for det medlem der stod for at rette alle de andres Latexfejl, hvilket kunne være meget tidskrævende. Gruppen valgte som løsning 
at et mere medlem skulle specialisere sig i Latex, og dermed også have overblik over projektrapporten og dokumentation. På denne måde sikredes det at der var 
styr på projekt dokumentation, et område som havde været plaget af manglende overblik. Gruppen lærte her at hvis der skal inddrages et nyt værktøj i et 
semesterprojekt, så skal der sættes tid af til at alle kan sættes sig grundigt ind i hvordan det virker. Man kan ikke satse på at det fungere hvis kun enkelte 
har et godt kendskab til værktøjet. 

\subsubsection{Slack}
Som kommunnikationsværktøj brugte gruppen slack. Dette værktøj kan integreres med andre værktøjet som github og google kalender, og blev brugt til 
sygemelding, mødeindkaldelser, generelle spørgsmål og beskeder fra scummaster. Det var allerede blevet brugt af nogle af gruppens medlemmer, og det var 
også disse som i starten brugte det mest. De øvrige medlemmer tog lidt tid om at komme igang, og det førte til noget frustration over manglende kommunikation.
Især scrum master var ret irriteret over manglende feedback på de beskeder og spørgsmål han lagde op, og det blev dog også hurtigt taget op på et retrospektiv.
Her blev det stillet som krav, at alle skulle downloade slack-appen til mobilen, og desuden skulle der markeres når man havde læst en besked. 
Scrum master fik desuden sin egen kanal på slack, så hans beskeder ikke forsvandt i mængden, og de andre medlemmer viste hvor de skulle finde de vigtigste 
beskeder. Slack fungerede i nogle perioder fint, men der var også perioder hvor folk ikke var online så ofte. Så et super effektivt beskedsystem blev det aldrig, 
men det var godt til have et sted hvor man kunne stille spørgsmål, og i øvrigt give nogle fælles beskeder.

I starten af projektet var alle medlemmer meget motiverede og villige til at påtage sig opgaver i projektet. Dog var der for meget fokus på implementeringsfasen,
hvilket satte gruppen bagud i forhold til designfasen.
Hele scrum metoden blev også diskuteret meget, og der var en del misforståelser i starten mht. de forskellige roller m.m. Efter de første par sprints
fik gruppen godt styr på scrum og dens værktøjer, og samarbejdet fungerede godt internt i gruppen.

Midvejs i projektet fungerede stå-op møderne stort set ikke, da gruppemedlemmer, uden at melde afbud, ikke dukkede op til møderne. Stå-op møderne
blev tydeligvis ikke priorieret særlig højt, hvilket også betød at kommunikationen internt i gruppen blev dårligere. Generelt var der på dette tidspunkt i forløbet
dårlig kommunikation, og manglende engagement fra visse gruppemedlemmer, hvilket smittede af på resten af gruppen.
Der var dog stadig en rimelig god fremmøde til arbejdsmøderne om mandagen. Det var på disse møder at der kom lidt samling på gruppen, og et overblik over projektet
kunne skabes. Arbejdsmøderne om onsdagen blev stort set ikke taget i brug længere, hvilket også var grundet i, at gruppen ikke havde så meget brug
for tværfagligt arbejde, som tidligere i projektet.

\section{Opnåede erfaringer}

\section{Fremtidigt arbejde}
