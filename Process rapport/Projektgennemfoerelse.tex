\chapter{Forord}
Dette er procesrapporten for semesterprojekt 3 på retningerne IKT, EE og E ved Aarhus University, School of Engineering. Rapporten er skrevet af gruppe 10, som har haft Søren Hansen som vejleder. Afleveringsdato for procesrapporten er 20. december 2016, og bedømmelse er den 18. januar 2017. Procesrapporten indeholder gruppens overvejelser og refleksioner omkring processen for semesterprojektet, og er skrevet på baggrund af de erfaringer gruppemedlemmerne har gjort sig under projektet.    

\chapter{Indledning}
Der er i dette semesterprojekt i modsætningen til tidligere projekter blevet lagt meget vægt på proccesen for projektet. Gruppemedlemmerne har derfor været nødt til at reflektere over hvilke procesmæssige udfordringer de er blevet stillet overfor, hvilket har været en ny måde at gribe projektet an på. Gruppen har især haft fokus på Scrum udviklingsmetoden, da det er første gang der bliver stiftet bekendtskab med denne metode i praksis. Denne udviklingsmetode gav gruppen en række nye værktøjer til styring af projekter, men har på nogle punkter været svær at implementere. I denne procesrapport vil der derfor være et særligt fokus på gruppens arbejde med Scrum som udviklingsmetode, og hvordan gruppens til- og fravalg af elementer fra Scrum påvirkede projektets forløb. Der vil også blive fokuseret på hvilke konflikter og problemer gruppen stødte på undervejs i projektet. Herunder hvilke beslutninger der blev taget for at løse disse udfordringer, samt gruppens refleksioner omkring disse.
\\
Tabel \ref{roller} viser ansvarsfordelingen af gruppens Scrum roller. \\

\begin{table}
	\centering
	\begin{tabular}{| l | c |}
		\hline
		Produktejer & Alle + vejleder\\
		\hline
		Scrum Master & Mikkel Busk Espersen\\
		\hline
		Supplerende Scrum master & Jacob Munkholm Hansen\\
		\hline
		Fast referent & Halfdan Vanderbruggen Bjerre\\
		\hline
		Team member & Alle\\
		\hline
	\end{tabular}
	\caption{Scrum rollefordeling}
	\label{roller}
\end{table}

\chapter{Projektgennemførelse}
\section{Gruppedannelse}
Allerede inden semesterstart, uge 35, var gruppen dannet. Fælles værdier og enighed om samarbejde på baggrund af erfaringer fra 2. semesters projekt blev grundlaget for dannelsen af gruppen.

Det har i processen omkring gruppedannelse været diskuteret, om gruppen bestod af nok medlemmer ift. det antal gruppemedlemmer ASE anbefaler til et 3. semesters projekt, nemlig 7-8 personer. Det kan være fordelagtigt at være en mindre gruppe, fordi der vil være færre synspunkter at tage hensyn til og 
dermed større mulighed for fælles enighed. En ulempe kunne være, at arbejdsbyrden per gruppemedlem bliver større. Disse overvejelser indgik i gruppens dannelse.\\

\section{Samarbejdskontrakt}
I starten af projektforløbet blev en samarbejdskontrakt udarbejdet, med henblik på at kunne skabe fælles rammer, som alle gruppens medlemmer var indforstået
 med. I gruppen havde alle erfaringer med brug af samarbejdskontrakter fra tidligere semestre. Under udarbejdelsen af samarbejdskontrakten havde gruppen mange gode og gavnlige diskussioner, som gav gruppen en 
 fælles forståelse af hvorledes gruppearbejdet skulle fungere. Efter udarbejdelsen af første udgave af samarbejdskontrakten, blev den ikke længere opdateret i 
 forhold til gruppens arbejdsmetoder. Generelt blev samarbejdskontrakten ikke taget i brug ved konflikter. I samarbejdskontrakten indgik der f.eks. et afsnit 
 omkring manglende fremmøde og konsekvenserne deraf. Man var som gruppemedlem forpligtet til at melde afbud ved udeblivelse og ligeledes meddele gruppen ved 
 forsinkelse. Dette blev overholdt den første periode af projektarbejdet, men senere blev dette afsnit ignoreret. I samarbejdskontrakten blev der beskrevet 
 hvilke konsekvenser en sådan forseelse bør munde ud i, men dette blev ligeledes heller ikke taget i brug. Som konklusion kan man sige at samarbejdskontrakten 
 som udgangspunkt gav gruppen et fælles grundlag for det fremtidige gruppearbejde, men da det var mest nødvendigt at bruge samarbejdskontrakten blev den 
 negligeret.
 
\section{Arbejdsfordeling}
Under projektarbejdet blev der taget en forholdsvis sikker tilgang til produktudviklingen. Til at starte med var der 4 fra IKT retningen, 1 fra E retningen, 
og 1 fra EE retningen. Derfor blev det hurtigt besluttet at de 4 fra IKT skulle beskæftige sig med softwareudvikling, mens den resterende gruppe skulle
beskæftige sig med hardwareudvikling. I starten af semesteret blev arbejdsfordelingen ændret, da en ud af de fire IKT’ere valgte at skifte retning, og dermed 
skifte over på hardwareudviklingsgruppen. Dog var der i slutningen af projektet blot 4 medlemmer tilbage i gruppen, hvilket betød at alle blev involveret i 
alle dele af produktudviklingen. 


\section{Planlægning}
Der blev lavet en samlet projektplan, men den blev stort set ikke benyttet.
Der skete mange ændringer undervejs i projektet og flere sprints blev forlænget på grund af manglende tid. En af de store udfordringer var at vurdere omfanget 
af forskellige opgaver i et sprint, og derfor endte gruppen ofte med ikke at blive færdige til sprintets afslutning. Det viste sig at en af de sværeste Scrum 
discipliner var at vurdere opgavernes tidsomfang. 

\section{Projektledelse og -administration}
Gruppen havde i sin fordeling af projektets ledelse som udgangspunkt fulgt de af Scrum definerede roller, hvis nærmere beskrivelse og realisering kan findes 
senere i rapporten i afsnittet Udviklingsforløb under Scrum. Grundet problemer i udførelsen af arbejdsopgaver, bl.a. lavt engagement fra visse 
gruppemedlemmer, blev der fra sprint 6 derfor udpeget en leder, der skulle sørge for at gruppen kunne færdiggøre projektet til tiden. Gruppen følte at denne
ændring var en nødvendighed for det videre arbejde, da der tydeligvis manglede den fornødne selvdiciplin
og ansvarsfølelse til at arbejde med en flad stuktur. Efter at gruppen var blevet reduceret til fire mand, blev den flade ledelsesstruktur dog atter anvendt.
\\
Til at administrere projektet blev der benyttet adskillige værktøjer. Disse inkluderede bl.a. \\
Scrum-værktøjet \textbf{Agile board}, hvor issues blev opslået og fordelt for hvert sprint; \\
\textbf{Logbog}, hvor der blev ført dokumentation for hvert medlem; \\
\textbf{Github}, hvor gruppen delte og holdte styr på sine dokumenter og filer; \\
\textbf{LaTeX}, hvori al udfærdigelse af rapport og dokumentation er blevet ført; \\
\textbf{Slack}, hvor den hovedsagelige kommunikation uden for faste mødetidspunkter er blevet holdt mellem gruppens medlemmer.

Generelt kan der siges om gruppens tilgang til disse værktøjer, som de fleste medlemmer manglede erfaring med, at der gennem projektforløbet har været en dalende interesse for at benytte sig af deres fulde funktionalitet, da et stigende arbejdspres efterlod mindre overskud for det enkelte medlem til at sætte sig grundigt ind i denne. Mere information herom og om værktøjerne generelt kan findes i bilag\footnote{Se bilag Værktøjers brug}.

\section{Møder}
Ifølge samarbejdsaftalen skulle gruppen afholde møder to gange ugenligt (mandag og onsdag), med henblik på at lave tværfagligt arbejde samt at plænlægge
de forskellige sprints. Derudover skulle der afholdes korte stå-op møder tre gange ugenligt for at opretholde god kommunikation internt i gruppen,
hvilket er en af Scrum metodens varetegn.
Det blev besluttet, at der hver mandag skulle afholdes arbejdsmøde fra 17-21, hvor alle gruppemedlemmer skulle deltage, og dermed sikre at der var
et tæt samarbejde på tværs af faglige kvalifikationer. Grunden til det sene tidspunkt var udfordringer mht. skemaet for de forskellige studieretninger,
samt individulle hensyn. 

I starten af projektet formåede gruppen at overholde de aftalte mødetidspunkter, dog blev stå-op møderne ofte uproduktive, og det blev diskuteret en del 
hvorledes de kunne gøres mere effektive. Enkelte gruppemedlemmer ønskede endda flere af disse korte møder for at få bedre kommunikation. Arbejdsmøderne 
fungerede godt, dog kom gruppen ofte til at bruge meget tid på diskussioner, der ikke var relevante for projektet. Problemet med stå-op møderne 
blev addreseret ved at hver gruppemedlem kort og præcist skulle komme med en mundtlig rapport, hvorefter der hurtigt kom et overblik over gruppemedlemmernes
 status. Herudover blev arbejdsmøderne effektiviseret ved at gruppen blev delt op i mindre enheder, som hver fik uddelt en delopgave. Gruppen blev hurtigt 
 enig om at arbejdsmøderne fra 17-21 var en dårlig ide, og dette blev hurtigt lavet om således at der blev arbejdet fra 14-18. Det andet faste arbejdemøde, 
 der lå om onsdagen, blev også hurtigt afskaffet, da der var brug for større fleksibilitet. Der lå ofte laboratorieøvelser fra andre fag på samme tidspunkt, 
 og derfor blev dette arbejdemøde kun taget i brug, når det var nødvendigt, og når det passede med skoleskemaet.    

\section{Udviklingsforløb} 

\subsection{Scrum}
\subsubsection{Konflikter med ASE-model}
Allerede inden projektets start havde gruppen lagt sig fast på at arbejde med Scrum og tilegnet sig de grundlæggende principper for denne udviklingsmetode. 
Der blev fra skolens side lagt op til at anvende Scrum jf. "Vejledning til udviklingsprocessen for semesterprojekt 3 v. 1.3". Dog blev der 3 uger inde i 
projektforløbet lagt en fornyet version op af denne vejledning, som ændrede rammerne for fleksibiliteten af udviklingsprocessen. Gruppens opfattelse af 
Scrum-forløbet skulle derfor op til revision, da Scrum i den fornyede vejledning blev påkrævet at følge ASE-modellens vandfaldsstruktur. Gruppens opfattelse 
af Scrum var en udviklingsmetode med meget frie rammer, hvor hvert sprint var en videreudvikling af prototypen med dokumentation og implementering som 
parallelt løbende opgaver. I stedet for dette skulle der følges en struktur, hvor ASE-modellens faser blev færdiggjort kronologisk. Specielt de påkrævede 
reviews gjorde det svært at følge Scrum, da der skulle tages mange forholdsregler for at kunne nå deadlines for disse reviews.
 
\subsubsection{Roller}
Product owner-rollen blev uddelegeret på hele gruppen, og alle var ansvarlige for oprettelse af issues på backloggen. Der var i starten af forløbet en vis 
usikkerhed om hvem, der skulle overtage denne rolle, da det medlem, som påtog sig denne, i princippet ikke samtidig kunne være en del af udviklingsholdet. 
Det blev af vejleder foreslået, at rollen kunne gå på tur for hvert sprint, men dette, gruppens størrelse og arbejdsfordelingen taget i betragtning, kunne 
ikke lade sig gøre, da det ville medføre, at pågældende medlem ikke kunne tage del i arbejdsprocessen. Eftersom der var et praktisk behov for, at hvert medlem
kunne bidrage sit til, at sprintet blev gennemført, blev det besluttet, at gruppen som helhed påtog sig ansvaret som dels product owner og udviklingshold. I 
løbet af og efter sprint 5 kom gruppen dog frem til den konklusion, at det kunne have været en fordel med en enkelt product owner, som havde et klart overblik 
over, hvad der skulle nås i projektet, da denne kunne have sat nogle klare mål for projektet og hvert sprint, i stedet for den hidtil fulgte orden, hvor 
gruppens medlemmer gennem forløbet jævnbyrdigt havde haft mulighed for at påvirke sprintets forløb. Sidstnævnte følger dog Scrum-metodens princip om 
fleksibilitet, men
projektets begrænsninger taget i betragtning (jf. forrige sektion) har dette i høj grad forvirret gruppen om, hvordan prioritering af opgaver skulle gribes an.

Scrum master-rollen blev påtaget af et enkelt medlem i starten af projektet. Da hverken denne eller andre medlemmer af gruppen havde nogen tidligere erfaring 
med Scrum, opstod der tidligt i forløbet misforståelser omkring Scrum masterens ansvar, så denne i princippet fik en typisk gruppeleders opgaver, hvilket 
skabte en hierarkisk opdeling i gruppen. Efter klargøring på et Sprint retrospective blev disse roller dog uddelegeret på hele gruppen. Således blev der 
oprettet en flad struktur i gruppen, hvor alle medlemmer var aktive beslutningsdeltagere i stedet for blot det ene medlem. Denne påtog sig dog stadig 
hovedansvaret med at holde gruppen
OBS på især mødeindkaldelser, men det, at resten af gruppen har måttet sætte sig grundigt ind i Scrum, har været en konstruktiv læringsoplevelse, som har
bidraget til et positivt arbejdsmiljø.

\subsubsection{Sprints}
Gruppen var fra starten af projektet opsat på at følge Scrums sprint-ordning. Allerede før det første sprint begyndte gruppen at tilføje opgaver til 
product-backloggen. Disse skulle dække over hele projektets forløb og være så kompakte, at medlemmet, der påtog sig opgaven, vidste præcist, hvornår opgaven 
var påbegyndt og afsluttet. Disse tidlige opgaver blev dog fjernet, da de viste sig ikke at blive relevante, da de forholdte sig for overfladisk til deres 
respektive område. Hvert efterfølgende sprint indledtes med et sprint planning-møde, hvor gruppen diskuterede, hvad der skulle arbejdes på i løbet af sprintet.
 På det første sprint planning-møde begik gruppen de fejl, af mangel på bedre viden: at uddelegere opgaver til medlemmer med det samme; ikke at have et 
 egentligt formuleret mål med sprintet; at påtage sig en lineær fremgangsmåde til de enkelte issues, således at en lang række opgaver nødvendigvis krævede, 
 at en tidligere opgave var blevet løst.

Efter samtale med vejleder indstillede gruppen sig på, at følge en mere Scrum-rettet tilgang. Dette indebar blandt andet at formulere et mål for et sprint, at 
lave en risikoanalyse og lade gruppens medlemmer påtage sig opgaver løbende under et sprint. I starten af andet sprint blev gruppen dog klar over, at der skulle 
foretages review på visse arkitekturrelaterede emner mm. Dette måtte prioriteres over den frie struktur, Scrum tilbyder, da disse reviews, som tidligere nævnt, 
var påkrævet for projektets gennemførelse. Det blev dog af gruppen bestemt, at Scrum skulle indkapsle de fornødne aspekter af ASE-modellen, således at der 
stadig blev opstillet opgaver, som gruppens medlemmer frit kunne påtage sig, og at der blev sat et mål for sprintet: At blive færdig med systemarkitekturen. 
Gruppens samtlige medlemmer oplevede styrken ved Scrum: Ved på agile-boardet manuelt at afslutte en opgave fik det enkelte medlem en 
bekræftelse på, at dennes arbejde gavnede projektets fremgang, og denne følte sig da også motiveret til at påtage sig en ny opgave. Dette sprint var 
højdepunktet for gruppens arbejde med Scrum. Tredje sprint fortsatte efter samme princip. Dette forløb sig forholdsvist glidende, men i forbindelse med
udførslen af 
applikationsmodellerne, og muligvis tidspresset af reviewet, blev visse opgaver for dette sprint ikke fyldestgørende udført. Bl.a. interne stridigheder i 
gruppen samt forvirring omkring, hvilken retning projektet skulle gå, førte til, at de næste to sprints kulminerede i en total omstrukturering af 
projektadministrationen. Under disse to sprints har der været en tendens til, stik imod Scrum's principper, at udvide sprintets forløb, påføre sprint-backloggen
nye opgaver under sprintet og generelt at 
ignorere opdatering af arbejdsproces jf. logbog. Dette kan forklares ved, at der jf. ASE-modellen af gruppen har været et fokus på at nå visse rettelser af 
systemarkitektur og at gøre klar til andet review, så der opstod en følelse af, at arbejdsbyrden blev umådeligt forøget ved at holde sig principfast til Scrums
 midler. Dette, samt
et stærkt behov for at danne sig en ren retningsplan for projektets fremtidige forløb mod projektets deadline, foranledigede gruppen til at sætte Scrum til 
side til fordel for en mere lineær arbejdsmodel, hvor der blev sat fokus på, at indskrænke projektet til at få lavet en færdig prototype og skrevet rapport og 
dokumentation uantastet, at det skred imod Scrum.


\subsection{Gruppedynamik}
\subsubsection{Gruppefølelse}
Som nævnt blev gruppen dannet allerede inden semesterstart, og der var en opfattelse af gruppen som "stærk", med engagerede medlemmer der var fagligt dygtige. 
Dette gav gruppen et godt udgangspunkt for et vellykket semesterprojekt, hvilket bidrog til en god stemning, og et godt sammenhold. Dette fortsatte i de første
 3 sprints, hvor der var god mødediciplin og hvor gruppen var samlet omkring de opgaver der skulle løses. Gruppen mødtes jævnligt til stå-op/arbejdsmøderne, 
 hvilket også bidrog til at gruppen følte sig samlet omkring de udfordringer gruppen stod overfor. Dog skete 
der et 
stort skift i gruppedynamikken efter efterårsferien, og i de følgende sprints, 4 og 5, opstod der nærmest en opløsning af gruppefølelsen. Kommunikationen 
begyndte at blive dårlig, og visse gruppemedlemmer mødte ikke op til stå-op møderne, hvilket smittede af på de andre medlemmer. Til sidst var gruppen stort 
set splittet op i mindre enheder der hver tog sig af sit eget. Mellem software- og hardwaregruppen var der ingen 
indbyrdes kommunikation eller overblik over projektet som helhed. Gruppen var dårlig til at håndtere disse problemer, og i flere uger blev denne situation 
uændret, hvilket havde en negativ effekt på gruppefølelsen. Dette blev heller ikke bedre af at et gruppemedlem forlod gruppen. Det var først i sprint 5, at 
disse problemer blev taget op med vejleder, og det blev besluttet at der skulle føres en stram linje mht. mødedeltagelse på stå-op møderne. Der blev i det 
hele taget rusket godt op i gruppen af Scrum master, og de resterende medlemmer begyndte at komme på rette spor igen. Efter endnu en person forlod gruppen,
var der kun de medlemmer tilbage, som ydede en aktiv insats, og der kom et bedre sammenhold i gruppen. Set i bagspejlet skulle der klart have 
været taget hånd om den manglende gruppefølelse tidligere. 

\subsubsection{Effektivitet og struktur}
For gruppen har struktur gennem hele forløbet været afgørende for effektiviteten. Den skiftende styringsform har ført til diskussioner
 om løsninger på problemerne der er opstået, og har derigennem fået gruppen til at arbejde mod bedre vilkår. I det store hele er det lykkedes gruppen at bevæge
 sig fremad og udvikle sig til det bedre. Med en konstant revurdering af sig selv har gruppen formået at forstå hinanden og processen markant bedre.
Et stort problem har været den flade struktur uden en ekstern product owner. Uden eksterne krav havde gruppen kun ansvar for sig selv hvilket resulterede i en 
doven og passiv holdning til projektet. Det sås bl.a. under sprint 4 og 5 hvor sprintene blev udskudt fordi opgaverne ikke var blevet løst indenfor den 
normerede tid. I stedet for at afslutte  sprintene, holde et grundigt analyserende og diskuterende retrospektiv og starte på en frisk, blev sprintene 
forlængede med det argument at det var samme opgaver som skulle løses. Denne beslutning resulterede dog i et mistet overblik og en demotiverende udvikling 
samt et meget lavt effektivitetsniveau. Under denne fase fik gruppen en endnu større udfordring - to medlemmer valgte at melde sig ud af gruppen.
 Det efterlod 4 mand til at gennemføre et projekt der i forvejen syntes ambitionsfyldt. Det giver anledning til at tro at gruppen faldt yderligere fra hinanden,
 men der skete en opblomstring mellem medlemmerne. Det samlede gruppen og der blev gjort tiltag for at effektivisere - deadlines/sprints på under en uge. 
 Samtidig blev det aftalt at et medlem ville blive noteret hvis denne ikke havde færdiggjort opgaver, som personen var blevet
 pålagt, og vedlagt i den endelige aflevering. Tiltagene øgede motivationen og havde den ønskede effekt.

\subsubsection{Konflikthåndtering}

\subsubsection{Eksempel på konflikt}
Gruppen havde ofte vidt forskellige opfattelser af, hvordan en opgave skulle løses. Mange gange blev beslutningerne taget på stedet, fordi et gruppemedlem var
 bedre til at argumentere for sit synspunkt end et andet. Dette resulterede i mange forhastede beslutninger, som senere i projektarbejdet skulle laves om. 
 Under udarbejdelse af applikationsmodeller, opstod den første konflikt som blev taget op på gruppebasis. Konflikten bestod i at flere fra gruppen havde 
forskellige tanker omkring, hvordan et sekvensdiagram skulle udarbejdes. Det betød at der blev udarbejdet et sekvensdiagram, som mange i gruppen var uenige i. 
Her blev vejleder for første gang brugt til at håndtere en konflikt i gruppen. Der blev aftalt med vejleder at parterne skulle udarbejde et forslag til 
et sekvensdiagram, og der derefter skulle udvælges den rette til projektet. Dette fungerede som en rigtig løsning, da det endelige sekvensdiagram blev udvalgt 
af hele gruppen, og at der generelt var tilfredshed omkring det.

\subsubsection{Ambitionsniveau}
En anden konflikt som der ofte opstod i projektarbejdet, var at der i gruppen var forskellige ambitionsniveauer. Ofte havde gruppens medlemmer forskellige 
løsninger til samme problem, hvor en af løsningerne konstruktionsmæssigt krævede mere tid. Derfor har det i startfasen ofte været således at gruppen som 
udgangspunkt 
valgte den sværeste løsning, for derefter at nedjustere kravene, således at sværhedsgraden og arbejdsbyrden kunne tilpasses gruppen, og den tid der har været 
til rådighed. Efter reduktionen til fire mand, blev gruppen bedre til at jurstere sværhedsgraden af de enkelte opgaver.    

\subsubsection{Frafald}
En konflikt som har været meget afgørende for projektet har været, at to af gruppemedlemmerne meldte sig ud af projektarbejdet. Det startede med at 
gruppemedlemmerne undlod at komme til møderne uden afbud. Dette blev der ikke taget hånd om, og der blev ikke holdt nogen samtale for at adressere problemet.
 Derfor var der i gruppen i en lang periode en forventning om, at gruppemedlemmerne ville komme tilbage til gruppearbejdet. Derfor blev 
 arbejdsbyrden ikke nedjusteret i forhold til at gruppen nu kun var på 4 personer. Årsagen til denne konflikt var manglende dialog. Havde gruppen taget 
 dialogen, og fundet ud af hvad der skulle ske, ville gruppen ikke nå til dette problem.

\section{Opnåede erfaringer}

\section{Fremtidigt arbejde}
I fremtiden vil det være godt at sikre sig at en pause i skolegangen ikke bliver en hindring for projektet, og man kunne evt. have fortsat kommunikationen 
hen over ferien, og givet medlemmerne nogle lektier for. Desuden ville det måske være godt at holde et møde straks efter ferien for at sikre sig, at man har 
et godt udgangspunkt for det fortsatte arbejde.  