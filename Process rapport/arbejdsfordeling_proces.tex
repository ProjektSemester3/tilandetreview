\section{Arbejdsfordeling}
Under projektarbejdet blev der taget en forholdsvis sikker tilgang til produktudviklingen. Til at starte med var der 4 fra IKT retningen, 1 fra E retningen, og 
1 fra EE retningen. Derfor blev det hurtigt besluttet at de 4 fra IKT skulle beskæftige sig med softwareudvikling, mens den resterende gruppe skulle beskæftige 
sig med hardwareudvikling. I løbet af starten af semesteret blev arbejdsfordelingen ændret, da en ud af de fire IKT’ere valgte at skifte retning, og dermed 
sprænge på hardwareudviklingsgruppen. 
I og med at der i projektet blev brugt SCRUM, blev der selvfølgelig også oprettet opgaver, som de enkelte gruppemedlemmer kunne påtage sig. I løbet af de første 
sprint blev opgaverne uddelegeret ved oprettelse. Det vil sige at når en opgave blev oprettet, så blev person som skal løse opgaven også valgt. Efter et par 
sprint hvor der blev kørt med denne form for opgaveuddelegering, blev det besluttet at opgaverne blot skulle op på sprintbackloggen, således at de forskellige 
gruppemedlemmer kunne påtage sig opgaverne løbende. Dette fungerede meget fint, og sprintet hvor dette blev testet var succesfuldt. Senere i projektarbejdet 
følte gruppen, at der var behov for et større overblik, på grund af det store pres gruppen var under. Dette skyldes at en mand meldte sig ud af gruppen. Derfor 
blev det besluttet at opgaverne blev uddelegeret ved oprettelse, som det blev gjort i starten af projektarbejdet.
