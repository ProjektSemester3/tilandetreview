\section{Konklusion}
Formålet med projektet var at lave prototypen for en automatisk vinåbner. Denne prototype skulle være i stand til at detektere vinflaskens position, således at
åbningen af flasken kunne gennemføres. Desuden skulle prototypen kunne indstilles til at åbne en vinflaske på et forudbestemt tidspunkt, med henblik på at 
opnå en korrekt iltning af vinen.\\

Det lykkedes i nogen grad at få udviklet de enkelte delelementer til prototypen. Der blev gennemført succesfulde modultest af både motorer og sensorer til 
positioneringen. x/y/z motorer kunne bevæge den indre ramme, og sensorer detekterede flaskens placering i systemet.\\

Brugergrænsefladen blev implementeret med funktioner til åbning af vin, og planlægning af vinåbningen. Disse fungerede efter hensigten, og sendte de ønskede
kommandoer videre i systemet.\\

Skelettet for vinåbneren blev også konstrueret, og motorer, sensorer blev moteret herpå. Hvilket betød at de fysiske rammer til en fungerende prototype var opsat.

Åbningsmekanismen blev dog ikke designet, og prototypen var i sidste ende ikke i stand til åbne en vinflaske. Gruppen havde undervurderet kompleksiteten af denne
del af projektet, og måtte til sidst opgive at få det færdiggjordt. Dette var trods en ihærdig indsats, og et stort ønske om at få denne del af prototypen til
at fungere.

Den serielle kommunikation imellem de logiske enheder fungerede kun som envejs kommunikation. Det lykkedes at sende kommandoer fra brugergrænsefladen til
resten af systemet, men der kunne ikke læses kommandoer tilbage, hvilket gjorde at statusbeskeder til brugeren ikke var mulige. Fejlen er desværre ukendt i
skrivende stund.

Udfordringer med skellettet viste sig at give problemer for gennemførelsen af integrationstesten. Trods succesfulde modultest, kunne det samlede system ikke
bevæge x og y akserne tilfredstillende. Dette skyldes små skævheder i konstruktionen, der gjorde at motorerne ikke kunne trække de bånd som skulle flytte x/y
akserne.       

Trods en utilfredstillende integrationstest, gjorde de succesfulde modultest af systemets delelementer, at prototypen kan danne et godt fundament til videre 
udvikling af en automatisk vinåbner.