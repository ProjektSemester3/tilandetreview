\chapter{Integrationstest}

De implementerede delelementer for WinePrep blev samlet med henblik på at teste en samlet prototype. Denne prototype indeholdt
funktionalitet til bevægelse af x/y/z akserne, og detektering af en vinflaske. Herudover blev brugergrænsefladen forbundet til 
resten af systemet, hvorfra det var muligt at igangsætte disse funktionaliteter. \\

For at kunne teste denne prototype, var den fysiske rammer en nødvendighed. Dennes konstruktionen viste sig dog at have en del
ujævnheder pga. dårligt designede 3D-prints. Dette betød at motorerne for X/Y akserne ikke kunne trække de bånd som skulle 
bevæge sensorerne. \\

Nedenfor ses accepttest af hovedscenarierne for usecase 1 og 2. Udvidelser til usecasene er ikke medtaget. 

\chapter{Accepttest Specifikation}
\section{Test af Use-case 1}
\begin{table}[H]
	\centering
	\caption{Accepttestspecifikation \nameref{UC1}: Hovedscenarie}
	\label{ATUC1:Hovedscenarie}
	%Se Tabel \ref{ATUC2:Hovedscenarie} på side \pageref{ATUC2:Hovedscenarie}
	\begin{tabularx}{360pt}{ TX  TX }\hline
		\rowcolor{white}	
		\textbf{Use-case under test} & \nameref{UC1} \\
		\rowcolor{lightgray}
		\textbf{Scenarie} & Hovedscenarie \\\rowcolor{white}	
		\textbf{Prækondition} &
		En gyldig type vinflaske er korrekt anbragt i maskinen og systemet er klar til brug. Desuden
		er vinflasken uåbnet og forseglingen er fjernet
 \\
		\hline
	\end{tabularx}
	\begin{tabularx}{360pt}{  p{26pt} TX  TX TX  TX}
		\textbf{Step} & \textbf{Handling} & \textbf{Forventet observation/resultat} & \textbf{Faktisk observation/resultat} & \textbf{Vurdering (OK/FAIL)}\\
		1 & Tryk på Åbn nu på
		brugergrænsefladen & Vinflasken låses og åbnes af
		systemet, og bruger
		meddeles via
		brugergrænsefladen
		om, at vinen er åbnet &
		Motorer til X/Y akserne igangsættes, men akserne bevæges ikke. Z akserne bevæges. Detektering af vinflaske ikke mulig. Ingen besked til bruger & FAIL \\
		\hline
	\end{tabularx}
\end{table}


\section{Test af use-case 2}

\begin{table}[H]
	\centering
	\caption{Accepttestspecifikation \nameref{UC2} Hovedscenarie}
	\label{ATUC2:Hovedscenarie}
	%Se Tabel \ref{ATUC2:Hovedscenarie} på side \pageref{ATUC2:Hovedscenarie}
	\begin{tabularx}{360pt}{ TX TX }\hline
		\rowcolor{lightgray}	
		\textbf{Use-case under test} & \nameref{UC2} \\
		\rowcolor{white}
		\textbf{Scenarie} & Hovedscenarie \\\rowcolor{lightgray}	
		\textbf{Prækondition} &
		En gyldig type vinflaske er korrekt anbragt i maskinen og systemet er klar til brug. Desuden er vinflasken uåbnet og forseglingen er fjernet \\
		\hline
	\end{tabularx}
	\begin{tabularx}{360pt}{ p{26pt} TX TX TX TX}
		\textbf{Step} & \textbf{Handling} & \textbf{Forventet observation/resultat} & \textbf{Faktisk observation/resultat} & \textbf{Vurdering (OK/FAIL)}\\
		1 & Tryk på planlæg åbning på brugergrænsefladen & Undermenuen planlæg åbning vises på brugergrænsefladen & Undermenuen vises & OK \\
		2 & Indstil på brugergrænsefladen klokkeslættet 2 timer frem & Den indstillede tid vises til det valgte klokkeslæt & Klokkeslættet vises & OK \\
		3 & Tryk på bekræft & Hovedmenuen vises og det valgte klokkeslæt vises i aktuel info på brugergrænsefladen & Det valgte klokkenslet vises i aktuel info & OK  \\
		4 & Vent 2 timer & Vinflasken er åben og systemet meddeler bruger via brugergrænsefladen om at vinen er drikkeklar & Vinflasken er ikke åben, og der er ingen beskeder til brugeren & FAIL  \\
		\hline
	\end{tabularx}
\end{table}