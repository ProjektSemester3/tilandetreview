\chapter{Forord}
Denne rapport er skrevet på 3. semester af gruppe 13, på retningerne IKT og EE ved Aarhus Universitet, Ingeniør højskolen. 
Vejleder for dette projekt er Søren Hansen. Afleveringsdatoen for denne projektrapport er den 20. December 2016, og bedømmelse er den 18. Januar 2017.
Rapporten er udarbejdet på baggrund af den dokumentation, som kan findes i bilaget for projektrapporten.

\section{Læsevejledning}
Det er tiltænkt at rapporten skal læses i kronologisk rækkefølge, dog kan afsnittene omkring implementering og test af delsystemerne læses
uafhængigt af hinanden. De forskellige dele er inddelt i kapitler. Hvert kapitel indeholder sektioner med dertil hørende undersektioner. Disse er alle nummererede.
Der vil blive brugt initialer på gruppens medlemmer til angivelse af, hvem rapportens sektioner er skrevet af: \\
\\
Mikkel Busk Espersen (MBS), \\
Jacob Munkholm Hansen(JMH), \\
Ahmad Sabah (AB), \\
Halfdan Vanderbruggen Bjerre(HVB). \\
\\
I de udarbejdede UML- og SysML-diagrammer og beskrivelser af disse vil der blive refereret til p- og s-motorer. Disse dækker over motorerne til styring af 
henholdsvis åbningsmekanismen(reference til ordliste) og skruen.

\section{Hovedansvarsområder}
Tabel xx viser fordelingen af hovedansvarsområder for produktet fordelt på gruppemedlemmer. Emnerne er inddelt i primær og sekundær, som informerer om 
medlemmers specialistviden og kernekompetencer indenfor produktudviklingen. Enkelte sekundære felter er tomme, dette betyder at ingen har været sekundær på 
emnet.\\

\begin{tabular}{| l | c | c |}
\hline
Emne & Primær & Sekundær\\\hline
Brugergrænseflade (GUI) & AS & HVB\\\hline
SPI DevKit-PSoC & HVB & JMH\\\hline
SPI PSoC-PSoC & HVB, JMH & \\\hline
PSoC software sensor & JMH & MBE\\\hline
PSoC software sensor & JMH & MBE\\\hline
Bipolære motorer & MBE & JMH\\\hline
Unipolære motorer & MBE & JMH\\\hline
DC motor & MBE & \\\hline
Konstruktion og mekanik & AS & HVB\\\hline
\end{tabular}