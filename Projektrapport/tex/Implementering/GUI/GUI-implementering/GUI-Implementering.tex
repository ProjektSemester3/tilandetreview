\subsubsection{GUI}

Programmet QT blev valgt da der tidligere har været arbejdet med QT i forbindelse med andre semesterprojekter. 

Sproget som brugergrænsefladen er skrevet i er C++ da det er dette sprog som teamet har haft størst erfaring med.

For at kommunikere med SPI driveren som ligger på Devkit8000 er funktionerne fread() og fwrite() brugt. 

PSOC er tidligere i koden blevet defineret som path’en på PSoC driven. I koden for read og write funktioner kan man se at OPEN\_BOTTLE, skrives til PSoC driveren ved hjælp af fwrite(). OPEN\_BOTTLE er tidligere blevet defineret som 5, hvilket i den anvendte protokol betyder at vinen skal åbnes. Det er samme kode der bruges til funktionen ”Planlæg åbning”. 

For læsning fra PSoC'en er funktionen fread() benyttet. Da der kan gå en del tid inden systemet får noget tilbage fra PSoC'en foregår læsningen i en tom for lykke som der kun kan brydes ud af hvis en af de 3 tideligere definerede svar modtages. Til visning af besked på brugergrænsefladen er QT's MessageBox klasse benyttet. 

For at få tiden talt ned og samtidigt displayet på skærmen er der blevet benyttet threads. Implementeringen af dette kan ses i dokumentationsbilaget x.
