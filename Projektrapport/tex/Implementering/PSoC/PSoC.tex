\subsubsection{Motor-/sensorstyring}
Klasserne fra klassediagrammet blev implementeret i form af hver deres headerfil efter princippet om høj samhørighed - lav kobling. \textit{main}-funktionen blev formet som en state-machine, der vha. en switch påkaldte de metoder, der skulle udføres på et givent tidspunkt i eksekveringen af programmet i henhold til sekvensdiagrammet (figur \ref{SD_PSoC}). Disse switches' cases bestemtes ud fra de kommandoer/beskeder, de enkelte PSoC's modtog fra hinanden eller DevKit8000. Der er ligeledes blevet oprettet en seperat \textit{status}-fil, som indeholder adskillige kommandoer/beskeder og forkortelser, der bruges igennem programmet, for at holde koden overskuelig og øge læsbarheden af denne.

\myparagraph{Hardware-grænseflade} \\
\\
\textbf{Sensorer} \\
Til at måle sensorerne benyttedes en SAR-ADC, som, efter hvert sample, returnerede en værdi i \textit{counts}. For at sammenligne med (GRAF FRA SENSOR-DATABLAD) konverteredes disse værdier til mV. Dermed kunne der fastslås, hvor vidt en flaske var registreret, ud fra afstanden forbundet med den målte spænding.

\textbf{Motorer} \\
Motorerne styredes vha. et PWM-signal, som gik fra en given GPIO-pen ud til STEP-inputtet på A4988-driveren, som talte et step op for hver rising edge på PWM-signalet, samt to digitale signaler, der gik til henholdsvis ENABLE- og DIRECTION-inputtene på driveren. \\
\\
\textbf{Knapper} \\
Knapperne implementeredes som interrupts der trigger på rising edge. Disse var hovedårsagen til, at der skulle min. 2 PSoS's, da hvert interrupt optager en port på PSoC'en, som kun har 6 til rådighed, mens der var behov for 8.