\subsection{SPI}

Til kommuniktaion mellem systemet CPU'er skulle der benyttes en serial protokol til afsendelse og modtagelse af databits.
Både Devkit8000 og PSoC understøtter UART, I2C og SPI. Gruppen tænkte i første omgang på at anvende UART, da kendskaben til denne protokol var god.
Det viste sig dog at UART porten på Devkit8000 bruges af anden hardware, og derfor stod gruppen tilbage med enten I2C eller SPI. Der blev under flere 
laboratorie øvelser på 3 semester anvendt SPI, og derfor var der allerede en SPI linux device driver tilgængelig. Det blev derfor besluttet at systemet 
skulle anvende SPI til Devkit8000-PSoC forbindelsen. I2C blev holdt åben som en mulighed til PSoC-PSoC forbindelsen i tilfældet af at der skulle kobles 
flere PSoC enheder på hinanden. I2C tillader nemlig flere enheder at være sammenkoblet med relativt få forbindelser, hvorimod SPI kræver en ny forbindelse for 
hver ny enhed der tilkobles. Systemet endte dog med kun at benytte to PSoC enheder, og da kendskaben til SPI var bedre, blev SPI også brugt til PSoC-PSoC
forbindelsen.     