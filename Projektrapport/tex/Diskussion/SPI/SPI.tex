\subsection{Seriel kommunikation}
\subsubsection{DevKit8000-PSoC Master}

Som det ses på figur \ref{LA_SPI_2}, blev der succesfuldt sendt de korrekte databits ud på SPI MOSI, SS gik også lav ved dataoverførelse hvilket er meget vigtigt
for at sikre, at den korrekte slave i vores tilfælde "PSoC Master" modtager kommandoerne. CLK ser rigtig ud med den clock mode som er specificeret.
Dette ses ved at CLK starter høj og skifter bits på nedafgående flanke, og læser på opadgående flanke. Dette svarer til CPHA = 1 og CPOL = 1, hvilket også var 
den opsæninger der er lavet for denne SPI forbindelse.\\

Problemer med MISO forbindelsen betød at projektets prototype ikke kunne sende status beskeder til brugeren, hvilket naturligvis var en stor skuffelse for
gruppen. Problemet ligger højst sandsynlig på selve PSoC, da der ikke læses noget data på MISO forbindelsen, og det derfor ikke har noget med DevKit8000 af gøre.
I test af GUI blev dette problem dog løst med en virtuel klasse, som fungerede som en stub og dermed simulerede SPI forbindelsen.


\subsubsection{PSoC Master - PSoC Slave}

Her ses igen er det er de rigtige databits som bliver sendt, og SS går ligeleder lav som den burde. Clock mode passer også fint med 
CPHA = 0 og CPOL = 0, som den var sat til. CLK starter lav som den skal, og skrifter på nedadgående flanke og læser på opadgående. Dette er helt som forventet. 
Og som det ses på figur \ref{LA_SPI_3}, så udskrives de korrekte bits til terminalen på PC, hvilket igen betyder at PSoC Slave modtager de korrekte bits, 
og derfor burde denne forbindelse virke fint.