\subsection{Diskussion af resultater}
\subsubsection{Motorer og sensorer}
\label{sec:HW_diskussion}
Den interessante test for motorer er den der viser forskellen i moment for uni- og bipolær motor. Ifølge Biot-Savarts lov burde den bipolære motor have forøget momentet med to gange, men resultatet viser sig at være betydeligt bedre, nemlig 18\%. Årsagen til dette skyldes højst sandsynligt motorens opbygning, men uden nærmere indsigt i denne vil der ikke her kommenteres yderligere på dette. Den manglende indsigt skyldes det mangelfulde datablad for 28BYJ-48, som gør det umuligt at kende detaljerne om motoren. Desuden undervises der ikke i denne transformation fra uni- til bipolær motor hvilket har gjort det svært at søge viden i kursernes indhold.

De imponerende resultater må dog siges at være en succes idet teorien underbygges ud fra facit af omdannelsen fra uni- til bipolær motor. 
\\
\\
Desværre formåede motoren ikke at drive akserne helt uden problemer, og det kan derfor heller ikke dokumenteres at disse motorer er egnet til opgaven.
\\
\\
Resultaterne fra sensorerne er generelt en succes idet de lever op til deres formål. Selvom præcisionen er relativ unøjagtig viser tests at motorerne reagerede på sensorernes detektering ved at ændre omdrejningsretning. Desuden er det ikke dokumenteret at sensorens ringe nøjagtighed har påvirket systemets præcision som helhed.