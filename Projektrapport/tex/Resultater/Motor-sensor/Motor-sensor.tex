\subsection{Motorer og sensorer}
Ved testen som set på figur \ref{fig:HW_bipolar_akse} formåede den bipolære motor at drive aksen med små ophold.
\\
\\
Udregningerne for momentet er foretaget med forskriften som ses i ligning \ref{eq:forskrift_moment}, hvor Tau er momentet, F er kraft, L er afstand i cm og sin(theta) altid er lig 1 fordi sinus til vinklen til 90 grader er lig 1.\\

\begin{equation} \label{eq:forskrift_moment}
	\tau = F \cdot L \cdot sin(\Theta)
\end{equation}

\noindent
Tabel \ref{tab:motor_moment} viser forskellen i moment for henholdsvis uni- og bipolær motor. Den bipolære motors moment er 118\% mere end den unipolæres. Resultaterne er udregnet ud fra testen med stativet fra figur \ref{fig:HW_stativ_test}. Forskellen er regnet ud ved nedenstående ligning.

\begin{equation} \label{eq:forskrift_forskel}
	forskel = \frac{MAX - MIN}{MAX} \cdot 100\%
\end{equation}

\begin{table}[H]
  	\centering
	\begin{tabular}{ |l|c|c| }
  		\hline
  		\textbf{28BYJ-48} & Unipolær & Bipolær\\
  		\hline
  		Moment i full-step mode & 363 gcm & 792 gcm \\
  		\hline
	\end{tabular}
	\caption[]{Moment for 28BYJ-48\footnotemark}
	\label{tab:motor_moment}
\end{table}
\footnotetext{Se 28BYJ-48Stativtest} 

\noindent
Figur \ref{Sensor_10cm} viser detektering af et objekt ved 10 cm afstand. Værdien 2241 er i mV. Figur 4 i databladet for sensoren viser ca. 2300 mV ved samme afstand hvilket giver en afvigelse på 2,57\% eller en unøjagtighed på 2,57 mm.

\begin{figure}[H]
	\includegraphics[scale=1]{tex/Test/Motor-sensor/Sensor_10cm.png}
	\caption{Resultat af detektering fra sensor ved 10 cm}
	\label{Sensor_10cm}
\end{figure}

\noindent
For flere resultater af målinger for afstand henvises til bilag Unipolær motor under afsnit for test af sensorer.