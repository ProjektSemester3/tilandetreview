\chapter{Indledning}
Interessen for robotteknologi er steget, især indenfor hjælpemidler til ældre. Den aldrende befolkninsgruppe striger stødt, og derfor er der behov for flere 
intelligente løsninger, som kan hjælpe fysisk hæmmede mennesker i deres hverdag. En af ideerne bag dette projekt var konstruktionen af en robot, som kunne hjælpe 
svagelige mennesker med at trække proppen ud af en vinflaske.\\

Smart-produkter er generelt blevet mere udbredte i moderne hjem, og der bliver større krav til hvilke daglige gøremål der skal kunne løses automatisk. Her kunne 
en automatisk vinåbner sætte nye standarder for smart-produkter i almindelige hjem. En sådan vinåbner kunne tilbyde en ny og innoverende måde at åbne en 
vinflaske. Med intelligente enheder som kan detektere vinflaskens postionen og mål, skulle vinåbneren åbne alle typer af vinflasker. \\

Et andet fokuspunkt for vinåbneren er forberedelse af vinen. For at få den optimale oplevelse ud af en vin, skal den åbnes rettidigt så den iltes før indtagelse.
Iltningstiden kan desuden variere fra vin til vin, og derfor kan uerfarne vindrikkere have svært ved at ilte deres vin korrekt. Dette kunne løses ved at 
automatisere denne iltningsprocess, hvor brugeren kan få åbnet vinen til et forudstemt tidspunkt bestemt ud fra vinens type.\\

Derudover kunne den automatiske vinåbner indeholde en række features som kunne forbedre vinoplevelsen. Dette kan gøre den til et tiltrækkende produkt også for 
vinentusiaster, som ønsker et premium produkt der kan give dem en større nydelse ved vindrikning.\\


\section{WinePrep}
Visionen for den automatiske vinåbner "WinePrep" var et system som kunne imødegå et hvert behov der måtte være indenfor drikning og forberedelse af vin. \\
Udover selve åbningen af en vinflaske skal systemet:\\
- Automatisk kunne finde vinflaskes top, så alle typer vinflasker uanset højde og øvrige mål kan åbnes.\\ 
- Kunne åbne vinen til et forudbestemt tidspunkt og derved sikre en optimal iltning af vinen\\
- Kunne måle vinens temperatur, og regulere denne så vinen kan nydes ved dens optimale betingelser.\\
- Indeholde en social medie platform "WineBook", hvor brugere kunne anmelde vine, og interagere med andre vinelskere\\
- Have en mobil applikation tilsluttet, der gør fjernbetjening af systemet muligt\\
- Kunne scanne etiketten på en vinflakse og finde information om vinen, herunder den optimale iltningstid, via en database\\
- Kunne dispensere korkproppen for vinflasken, efter vinåbningen er afsluttet.\\

Ved udformingen af det ideelle produkt er der ikke taget hensyn til gruppens begrænsede tid, ressourcer og kompetencer. Visionen for WinePrep fungere
i dette projekt blot som et startsted for det videre projektforløb. Ud fra det ideelle produkt vil gruppen udvælge de funktionaliteter, som vurderes til realistisk
set at kunne gennemføres.