\section{Metode}

\textbf{UML- og SysML}: Er værktøjer til konstruktionen af arkitekturen for systemet. De har til formål at give et visuelt overblik over de delelementer 
systemet består af. UML og SysML dækker over flere forskellige typer diagrammer til beskrivelse af software og hardware komponenter. \\ 

Til beskrivelsen af hardwarekomponenter og deres interne grænseflader kan \textbf {IBD og BDD} anvendes. BDD’et er brugt til at nedbryde systemet i blokke, 
således at man hurtig kan danne sig et overblik over hvilke fysiske elementer,systemet består af. IBD’erne er brugt til at beskrive de interne grænseflader der 
er i systemet. Altså ind- og udgangsportene som er på de forskellige dele af produktet.

Til beskrivelsen af softwarearkitekturen blev der udarbejdet \textbf{klasse- og sekvensdiagrammer}. Klassediagrammerne viser hvilke klasser systemet 
består af, mens sekvensdiagrammerne skal vise kommunikationen mellem disse klasser \\

Herudover indeholder projektrapporten \textbf{state mashines (STM) og flow charts} til beskrivelse af softwares adfærd.\\

\textbf{usecases}  er benyttet til at definere de funktionelle krav, mens \textbf{FURPS+} og \text{MoSCoW} er benyttet til at definere de 
ikke-funktionelle krav. Usecasene er udviklet ud fra systemniveauet.\\

\textbf{domænemodel} er ikke medtaget i rapporten, hvilket er en afvigelse fra ASE modellen. Da usecasene er lavet på systemniveau, er der ikke et godt 
fundament til udarbejdelsen af en domænemodel. Det er svært på systemniveau at udlede konceptuelle klasser for winePrep, og derfor giver domænemodellen ikke 
nogen værdi mht. applikationsmodellerne.\\
 
For at definere hvorledes software og hardware er allokeret, er der lavet \textbf{allokeringsdiagrammer}. Disse mapper hardware og software ned på de enkelte
hardware blokke.\\
