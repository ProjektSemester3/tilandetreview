\section{Diskussion - GUI}

Der er mange funktioner i brugergrænsefladen som til at starte med var tiltænkt, som ikke er blevet implementeret. Dette skyldes hovedsageligt 2 ting. Den første er at teamet ikke har haft den nødvendige erfaring til at kunne estimere et projekts omfang. Der var rigtig mange ting som blev planlagt som aldrig blev udført på grund af mangel på tid. Den anden store grund til at alle funktioner ikke kom med var at gruppen blev nedskåret til en 4 personers gruppe frem for en 8 personers grupper som projektet oprindeligt var tiltænkt for. Derfor er det naturligt at gruppen ikke kan nå lige så meget som en gruppe på 8 personer.

De test som blev udført på brugergrænsefladen var yderst succesfulde, da det ønskede resultat blev opnået. Under testen blev der forsøgt at sende kommandoen 5 ud igennem SPI, og dette lykkedes som det også fremgår af afsnittet Test.  Det var dog tænkt, og i første omgang implementeret således at brugergrænsefladen kan meddele brugeren meddelelse. Dette skulle ske, ved at den fik respons fra PSoC’en, og alt efter hvilken respons den fik, ville den frembringe en dialogboks. Dog virkede dette ikke da SPI driveren var ustabil, og der ikke ville læse. Det var kun muligt at skrive med SPI driveren i lange perioder.
